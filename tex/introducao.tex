% Comando simples para exibir comandos Latex no texto
\newcommand{\comando}[1]{\textbf{$\backslash$#1}}

Segundo dados da \citeonline{receita:dados-publicos:cnpj}, o Brasil possuía em abril de 2020 mais de 20 milhões de empresas ativas que realizam transações todos os dias envolvendo aquisição e transferência de mercadorias, prestação de serviços das mais diversas naturezas, transportes e devoluções. Essas empresas possuem uma série de obrigações contábeis e fiscais a cumprir todos os meses. Obrigações essas que se traduzem em documentos enviados ao governo contendo uma série de informações sobre cada transação efetuada pela empresa.



%%%%%%%%%%%%%%%%%%%%%%%%%%%%%%%%%%%%

% Este documento explica brevemente como trabalhar com a classe \LaTeX~\textit{icmc} para confeccionar trabalhos acadêmicos seguindo as normas da \sigla{ABNT}{Associação Brasileira de Normas Técnicas} e as \aspas{\textit{Diretrizes para apresentação de dissertações e teses da USP: documento eletrônico e impresso. Parte I (ABNT)}}, publicado pelo \sigla{SIBi}{Sistema Integrado de Bibliotecas} USP. O presente manual também atende as exigências prevista no regimento do Programa de Pós-graduação em \sigla{CCMC}{Ciências da Computação e Matemática Computacional} do \sigla{ICMC}{Instituto de Ciências Matemáticas e de Computação} da \sigla{USP}{Universidade de São Paulo}.


% A classe \textit{icmc} foi construída com base na última versão da classe \textit{abntex2} e do pacote \textit{abntex2cite}. Portanto, este documento exemplifica a elaboração de trabalho
% acadêmico (tese, dissertação e outros do gênero) produzido conforme a ABNT NBR
% 14724:2011 \textit{Informação e documentação - Trabalhos acadêmicos - Apresentação}.

% Assim, é altamente recomendável que seja consultada a documentação do \textit{abntex2}\footnote{http://abntex.net.br}. A classe \textit{abntex2} foi desenvolvida para facilitar a escrita de documentos seguindo as normas da ABNT no ambiente \LaTeX\;\cite{frasson:2005:classe_abnt}.

% Todo o trabalho de pesquisa e ajustes da presente classe \LaTeX~\emph{icmc} foram feitos pelo aluno mestrado do Programa de Pós-graduação em Ciência da Computação e Matemática Computacional, Humberto Lidio Antonelli, durante a confecção da sua monografia de qualificação.

% O requisito básico para utilização da classe \textit{icmc} é criar um documento desta classe com o comando
% \comando{documentclass[@parameters]\{icmc\}} e ter, no diretório de trabalho, o arquivo \emph{icmc.cls} presente. Entretanto, recomenda-se fortemente manter a estrutura de diretório inicial fornecida por este modelo. Além disso, para que o documento esteja em conformidade com as normas exigidas pelo programa de Pós-Graduação, o \textbf{projeto deve ser compilado utilizando \textit{XeLaTeX} ou \textit{LuaLaTeX}}. Esse processo de compilação é necessário para que as fontes externas utilizadas para gerar a capa sejam incluídas.

% Os parâmetros possíveis utilizados pelo \comando{documentclass} são:
% \begin{description}
% \item[qualificacao] Exclusivamente para monografias de qualificação em geral;
% \item[mestrado / doutorado] Identifica o curso ao qual o aluno pertence, sendo utilizado apenas uma das duas opcões disponíveis. O valor padrão é \textbf{doutorado};
% \item[pre-defesa / pos-defesa] Identifica a situação do documento (exceto para qualificação), sedo necessário apenas uma das duas opções. O valor padrão é \textbf{pos-defesa};
% \item[impressao] Gera exclusivamente uma versão para impressão do documento;
% \item[french, spanish, english, brazil] Adiciona o idioma para correta hifenização correta no documento. Os idiomas bases para o modelo (português e inglês) não precisam ser declarados.
% \end{description}
