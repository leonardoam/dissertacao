
\newword{Armazém de Dados}{ou \textit{Data Warehouse}, é um sistema de armazenamento estruturado de informações usado para a consolidação de informações e conceitos, geralmente envolvendo uma grande quantidade de dados. Esses dados são comumente armazenados em bancos de dados colunares pois envolvem o cálculo de agregações, e usam técnicas de Processamento Analítico em Tempo Real, também chamado \sigla{OLAP}{Online Analytical Processing}, para a coleta e tratamento de tais dados. Os dados são então usados em processos corporativos para a tomada de decisão, desenvolvimento e aperfeiçoamento de novos produtos, e até mesmo reaproveitados em aplicações}

\newword{\textit{Enterprise Resource Planning}}{ou \textit{ERP}, em tradução livre significa Planejamento de Recursos Empresariais, é um tipo de sistema para o gerenciamento de processos de negócio que podem atender diversas áreas, como financeiro, recursos humanos, compras e vendas, etc. No contexto de documentos fiscais, muitos desses sistemas possuem emissores capazes de gerenciar algumas das obrigações contábeis de forma automatizada}

\newword{Fonte Única de Verdade}{é uma prática adotada na construção de sistemas onde todos os dados conceitualmente relacionados a uma entidade importante para uma empresa ou organização são centralizados em uma única base de dados que passa a se tornar um banco de referência a outros sistemas, aumentando a consistência dos dados e diminuindo a carga cognitiva necessária para times indiretamente ligados a essa entidade}

\newword{\textit{Application Programming Interface}}{ou \textit{API}, em tradução livre significa Interface Programática de Aplicação, é um conjunto de ferramentas que permite a integração entre dois sistemas através de um protocolo de comunicação. Nesta integração são usadas definições comuns a ambos sistemas de modo a estabelecer um padrão de comunicação. APIs geralmente são construídas usando um protocolo de transferência de dados, como HTTP, REST, ou SOAP, e encapsulam o acesso a um banco de dados interno}
