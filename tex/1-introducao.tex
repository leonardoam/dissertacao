% Comando simples para exibir comandos Latex no texto
\newcommand{\comando}[1]{\textbf{$\backslash$#1}}

Segundo dados da \citeonline{receita:dados-publicos:cnpj}, o Brasil possuía em setembro de 2020 mais de 45 milhões de empresas ativas que realizam transações todos os dias envolvendo aquisição e transferência de mercadorias, prestação de serviços das mais diversas naturezas, transportes e devoluções. Essas empresas possuem uma série de obrigações contábeis e fiscais a cumprir todos os meses. Obrigações essas que se traduzem em documentos enviados ao governo contendo uma série de informações sobre cada uma dessas transações efetuadas.

A lei que obriga a emissão de notas fiscais é datada do ano de 1994~\cite{lei:1994:documentos-fiscais}. À época, a emissão desses documentos era feita em papel, o que causava grandes transtornos às empresas tanto por conta dos próprios processos envolvidos quanto por conta da guarda desses documentos.

outros trabalhos relacionados a estudo de documentos fiscais

outras analises do impacto economico da pandemia

Na seção~\ref{chapter:documenos-fiscais} será apresentado o domínio dos dados utilizados neste trabalho com uma explicação sobre documentos fiscais, a legislação brasileira envolvida, e os sistemas usados para captação dos dados. Em seguida, na seção~\ref{chapter:pandemia} será introduzido todo o contexto da Pandemia de COVID-19 e a sequência de eventos envolvida que, além de causar uma grave crise sanitária de dimensões mundiais, teve também seus impactos econômicos. A seguir, será descrita a base de dados utilizada com uma análise descritiva e detalhes do pré-processamento feito para a coleta dos dados na seção~\ref{chapter:base-de-dados}. Completar...
