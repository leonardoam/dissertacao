Em documentos acadêmicos podem existir as citações podem ser: \textbf{implícitas} quando as referências não fazem parte do texto ou \textbf{explícitas} quando o autor referente a citação é mencionado explicitamente na sentença. Nesse sentido, deve-se utilizar os comandos específicos para cada tipo de citação, ou seja, em citações explicitas deve-se usar o comando \comando{citeonline\{\}} e nas demais situações é usado o comando \comando{cite\{\}}. Alguns exemplos são apresentados no \autoref{qua:exemplo-citacao}.

\begin{quadro}[htb]
\caption{Exemplos de citações no documento}
\label{qua:exemplo-citacao}
\centering\small 
\begin{tabular}{|c|c|}        \hline
\textbf{Código em \LaTeX} & \textbf{Código Compilado}\\ \hline\hline
\begin{minipage}[t]{\VerbL}
\vspace{5pt}
\begin{verbatim}
A ironia será assim uma ... proposta
por \citeonline{10520:2000:4.1-1}.
\end{verbatim}
\vspace{5pt}
\end{minipage}
&
\begin{minipage}[t]{\LatL}
\vspace{5pt}
A ironia será assim uma ... proposta 
por \citeonline{10520:2000:4.1-1}.
\vspace{5pt}
\end{minipage}\\\hline

\begin{minipage}[t]{\VerbL}
\vspace{5pt}
\begin{verbatim}
\citeonline[p.~146]{10520:2000:4.2-2}
dizem que ... 
\end{verbatim}
\vspace{5pt}
\end{minipage}
&
\begin{minipage}[t]{\LatL}
\vspace{5pt}
\citeonline[p.~146]{10520:2000:4.2-2} dizem que {...}
\vspace{5pt}
\end{minipage}\\ \hline

\begin{minipage}[t]{\VerbL}
\vspace{5pt}
\begin{verbatim}
``Apesar das ... da filosofia''
\cite[p.~293]{10520:2000:4.1-2}.
\end{verbatim}
\vspace{5pt}
\end{minipage}
&
\begin{minipage}[t]{\LatL}
\vspace{5pt}
``Apesar das {...} da filosofia'' \cite[p.~293]{10520:2000:4.1-2}.
\vspace{5pt}
\end{minipage} \\ \hline

\begin{minipage}[t]{\VerbL}
\vspace{5pt}
\begin{verbatim}
Depois, ...  que prefiro
\cite{10520:2000:4.1-3}.
\end{verbatim}
\vspace{5pt}
\end{minipage}
&
\begin{minipage}[t]{\LatL}
\vspace{5pt}
Depois, {...} que prefiro \cite{10520:2000:4.1-3}.
\vspace{5pt}
\end{minipage}\\ \hline

\end{tabular}
\end{quadro}


Para especificar a página, seção ou capítulo consultado na referência é preciso acrescentá-lo entre colchetes com os comandos \comando{cite[página]\{\}} ou \comando{citeonline[página]\{\}}. O texto colocado entre colchetes aparecerá logo após o ano. Maiores informações sobre os comandos utilizados para citação posem ser consultados no manual de referência da abnTeX2, incluindo o uso de \textbf{apud} \cite{abntex2cite-alf}.


\section{Citações Indiretas}

As citações indiretas são caracterizadas como uma espécie de paráfrase das ideias de um determinado autor, ou seja, o pesquisador, por meio de suas próprias palavras, interpreta o discurso de outrem, contudo, mantendo o mesmo sentido. Outro aspecto que deve ser considerado é a necessidade de o autor (ou os autores) e o ano em que a obra foi publicada serem mencionados. 

Nas citações indiretas há duas formatações possíveis dependendo de como ocorre a citação no texto. Quando o autor é mencionado explicitamente utiliza-se o comando \comando{citeonline\{\}}, caso contrário, deve utilizar o comando \comando{cite\{\}}. 



\section{Citações diretas}
\label{sec-citacao}


As citações diretas ocorrem quando o texto de uma referência é transcrito literalmente. As citações diretas curtas (até três linhas) são inseridas no texto entre aspas duplas. As aspas simples são utilizadas para indicar citação no interior da citação: \aspas{Nas citações, as chamadas pelo sobrenome do autor [...] incluído na sentença devem ser em letras maiúsculas e minúsculas e, quando estiverem entre parênteses, devem ser em letras maiúsculas} \cite[sec.~5]{NBR10520:2002}.

\begin{verbatim}
``Nas citações, as chamadas pelo sobrenome do autor [...] incluído na 
sentença devem ser em letras maiúsculas e minúsculas e, quando 
estiverem entre parênteses, devem ser em letras maiúsculas''
\cite[5]{NBR10520:2002}.
\end{verbatim}

Cabe ressaltar que em \LaTeX as aspas iniciais são diferentes das finais. Para tanto, pode-se utilizar o comando \comando{aspas\{CONTEUDO\}} para inserir um determinado conteúdo entre aspas.

As citações diretas longas (com mais de 3 linhas) podem ser inseridas por meio do ambiente \texttt{citacao}:

\begin{citacao}
As citações diretas, no texto, com mais de três linhas, devem ser
destacadas com recuo de 4 cm da margem esquerda, com letra menor que a do texto
utilizado e sem as aspas. No caso de documentos datilografados, deve-se
observar apenas o recuo \cite[5.3]{NBR10520:2002}.
\end{citacao}

Use o ambiente assim:

\begin{verbatim}
\begin{citacao}
As citações diretas, no texto, com mais de três linhas [...] deve-se 
observar apenas o recuo \cite[5.3]{NBR10520:2002}.
\end{citacao}
\end{verbatim}

O ambiente \texttt{citacao} pode receber como parâmetro opcional um nome de
idioma previamente carregado nas opções da classe (\autoref{sec-hifenizacao}). Nesse
caso, o texto da citação é automaticamente escrito em itálico e a hifenização é
ajustada para o idioma selecionado na opção do ambiente. Por exemplo:

\begin{verbatim}
\begin{citacao}[english]
Text in English language in italic with correct hyphenation.
\end{citacao}
\end{verbatim}

Tem como resultado:

\begin{citacao}[english]
Text in English language in italic with correct hyphenation.
\end{citacao}
