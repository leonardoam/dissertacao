
Este capítulo encerra esse trabalho com as conclusões obtidas no decorrer do mesmo.

\section{Pontos positivos e negativos}

Este trabalho obteve sucesso em demonstrar o valor dos dados de documentos fiscais, obtendo análises significativas a partir de dados que muitas vezes são vistos apenas como uma burocracia pelas empresas.

A análise descritiva dos dados e do impacto econômico da pandemia apresenta também uma nova perspectiva do impacto de um evento tão importante, mesmo que estejamos lidando neste trabalho com uma amostra de dados em relação à economia brasileira.

A análise do impacto da pandemia do ponto de vista topológico usando redes complexas colabora com outras análises semelhantes, sendo que conseguimos alcançar resultados parecidos com trabalhos prévios, como a perda de conectividade, para grafos de diferentes complexidades e topologias.

O processamento de documentos fiscais possui também um desafio recorrente em trabalhos analíticos sobre eles que é o tratamento de inconsistências e incorreções nos dados. Apesar de ser um documento de alta criticidade jurídica, as informações presentes nos documentos nem sempre estão estruturadas de forma adequada e também podem apresentar problemas, o que pode causar ruído nas análises. Neste trabalho, apesar dos esforços de pré-processamento, foram encontrados alguns problemas do tipo e um importante desafio a ser ainda lapidado são novas formas de pré-processar esses documentos para melhor confiabilidade dos dados.

O estudo não conseguiu apresentar um classificador eficaz para identificar entidades afetadas pela crise, e também não apresentou análises mais granulares, como por exemplo sobre empresas.

\section{Próximos passos}

Como sugestão para trabalhos futuros, existem outros documentos fiscais com valiosos dados que não foram utilizados aqui. Por exemplo, o Conhecimento de Transporte Eletrônico é um documento de grande valor para análises de cadeias de transporte, como trabalhos anteriores já o fizeram. Outros documentos fiscais são portanto uma fonte rica de dados para futuros trabalhos.

Neste trabalho não foram utilizadas ferramentas de processamento distribuído que podem auxiliar em análises de maiores quantidades de dados, e poderiam auxiliar por exemplo em uma análise de impacto por empresas.

Outro desafio interessante seria transformar as informações aqui encontradas, de métricas e topologias de redes complexas, em produtos a serem utilizados pelas empresas brasileiras, transformando burocracia em oportunidade.

O autor deste trabalho finaliza esperando que as análises e esforços aqui feitas sejam úteis no futuro em outros trabalhos.
