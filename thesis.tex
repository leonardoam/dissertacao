% ------------------------------------------------------------------------
% ------------------------------------------------------------------------
% ICMC: Modelo de Trabalho Acadêmico (tese de doutorado, dissertação de
% mestrado e trabalhos monográficos em geral) em conformidade com 
% ABNT NBR 14724:2011: Informação e documentação - Trabalhos acadêmicos -
% Apresentação
% ------------------------------------------------------------------------
% ------------------------------------------------------------------------

% Opções: 
%   Qualificação          = qualificacao 
%   Curso                 = doutorado/mestrado
%   Situação do trabalho  = pre-defesa/pos-defesa (exceto para qualificação)
%   Versão para impressão = impressao
\documentclass[mestrado, pre-defesa]{packages/icmc}

% ---------------------------------------------------------------------------
% Pacotes Opcionais
% ---------------------------------------------------------------------------
\usepackage{rotating}           % Usado para rotacionar o texto
\usepackage[all,knot,arc,import,poly]{xy}   % Pacote para desenhos gráficos
% Este pacote pode conflitar com outros pacotes gráficos como o ``pictex''
% Então é necessário usar apenas um dos pacotes conflitantes
\newcommand{\VerbL}{0.52\textwidth}
\newcommand{\LatL}{0.42\textwidth}
% ---------------------------------------------------------------------------


% ---
% Informações de dados para CAPA e FOLHA DE ROSTO
% ---
% Tanto na capa quanto nas folhas de rosto apenas a primeira letra da primeira palavra (ou nomes próprios) devem estar em letra maiúscula, todas as demais devem ser em letra minúscula.
\tituloPT{Estudo do impacto econômico de uma pandemia utilizando dados de Documentos Fiscais}
\tituloEN{Study of the economic impact of a pandemic using Invoices}
\autor[Miguel, L. A.]{Leonardo Alves Miguel}
\genero{M} % Gênero do autor (M = Masculino / F = Feminino)
\orientador[Orientador]{Prof. Dr.}{Francisco Aparecido Rodrigues}
%\coorientador{Prof. Dr.}{Fulano de Tal}
\curso{MECAI}
\data{28}{01}{2021} % Data do depósito
\idioma{PT} % Idioma principal do documento (PT = português / EN = inglês)
% ---


% ---
% RESUMOS
% ---

% Resumo em PORTUGUÊS
% conter no máximo 500 palavras
% conter no mínimo 1 e no máximo 5 palavras-chave
\textoresumo[brazil]{
    Um resumo
    }{Algumas, palavras, chave}


% resumo em INGLÊS
% conter no máximo 500 palavras
% conter no mínimo 1 e no máximo 5 palavras-chave
\textoresumo[english]{
    A abstract
    }{Some, keywords}


% ----------------------------------------------------------
% ELEMENTOS PRÉ-TEXTUAIS
% ----------------------------------------------------------

% Inserir a ficha catalográfica
\incluifichacatalografica{tex/pre-textual/ficha-catalografica.pdf}

% DEDICATÓRIA / AGRADECIMENTO / EPÍGRAFE
\textodedicatoria*{tex/pre-textual/dedicatoria}
\textoagradecimentos*{tex/pre-textual/agradecimentos}
\textoepigrafe*{tex/pre-textual/epigrafe}

% Inclui a lista de figuras
\incluilistadefiguras

% Inclui a lista de tabelas
\incluilistadetabelas

% Inclui a lista de quadros
\incluilistadequadros

% Inclui a lista de algoritmos
\incluilistadealgoritmos

% Inclui a lista de códigos
\incluilistadecodigos

% Inclui a lista de siglas e abreviaturas
\incluilistadesiglas

% Inclui a lista de símbolos
\incluilistadesimbolos

% Dependências inseridas pelo autor
\usepackage{multirow}

% ----
% Início do documento
% ----
\begin{document}
% ----------------------------------------------------------
% ELEMENTOS TEXTUAIS
% ----------------------------------------------------------
\textual

\chapter{Introdução}
\label{chapter:introducao}
% Comando simples para exibir comandos Latex no texto
\newcommand{\comando}[1]{\textbf{$\backslash$#1}}

\section{Documentos Fiscais Eletrônicos}
\label{introducao:dfe}

Segundo dados da \citeonline{receita:dados-publicos:cnpj}, o Brasil possuía em setembro de 2020 mais de 45 milhões de empresas ativas que realizam transações todos os dias envolvendo aquisição e transferência de mercadorias, prestação de serviços das mais diversas naturezas, transportes e devoluções. Essas empresas possuem uma série de obrigações contábeis e fiscais a cumprir todos os meses. Obrigações essas que se traduzem em documentos enviados ao governo contendo uma série de informações sobre cada uma dessas transações efetuadas.

A lei que obriga a emissão de notas fiscais é datada do ano de 1994~\cite{lei:8846:documentos-fiscais}. À época, a emissão desses documentos era feita em papel, o que causava grandes transtornos às empresas tanto por conta dos próprios processos envolvidos quanto por conta da guarda desses documentos, que a depender do porte e do volume de transações das empresas exigia um grande esforço de armazenamento de milhares de documentos todos os meses. Era comum nas empresas a presença arquivos de aço para a guarda desses documentos.

A modernização da legislação que aconteceu na década seguinte possibilitou a diminuição desse problema, e através de parcerias entre os governos foi possível informatizar esses processos e o que era então papel passou a ser arquivo eletrônico armazenado em computadores ou servidores, viabilizando também a automação desses processos. Esse processo de modernização, ocorrido também em vários países da América Latina, é descrito por \citeonline{mello2014documentos} que relata uma convergência entre vários países para o uso de tecnologias semelhantes, como a estruturação dos dados em arquivos XML, a existência de uma versão impressa, como o \sigla{DANFE}{Documento Auxiliar da Nota Fiscal Eletrônica} e o \sigla{DACTE}{Documento Auxiliar do Conhecimento de Transporte Eletrônico}, o uso de assinaturas digitais para verificação de autenticidade e autoria, e o uso de \textit{web services} para a emissão e disponibilização desses documentos. \citeonline{pasa2001uso} descreve uma série de normas e recomendações que viabilizam o uso de documentos fiscais de forma confiável e legalmente viável e fala da necessidade de "transformar o contador em um verdadeiro analista de informação".

Outro importante ganho advindo da informatização desses documentos foi justamente a facilitação do uso desses documentos para não só cumprir obrigações contábeis e fiscais, mas também em processos internos de empresas e até mesmo análises financeiras e econômicas de diversas finalidades. Os \sigla*{DFe}{Documento Fiscal Eletrônico} documentos fiscais eletrônicos são ricos em informações, contendo dados detalhados de forma estruturada acerca das transações efetuadas entre empresas, onde são descritos produtos, valores, impostos, e classificações que possuem uma infinidade de aplicações.

Alguns trabalhos acadêmicos já exploraram anteriormente algumas dessas aplicações através do uso de documentos fiscais para propor soluções para a sociedade. \citeonline{fernandez2012avaliaccao} propõe o uso de documentos fiscais, como a \sigla{NFe}{Nota Fiscal Eletrônica}, o \sigla{CTe}{Conhecimento de Transporte Eletrônico}, e do \sigla{MDFe}{Manifesto Econômico de Documentos Fiscais} para o rastreamento de cargas e veículos, os tornando importante ferramentas de prevenção e combate à sonegação e fraudes. O trabalho de \citeonline{madeira2015aplicaccao} também propõe um método de mineração de texto para a detecção de fraudes e incorreções através de dados de Notas Fiscais de Serviços Eletrônicas, classificando-as quanto a essas incorreções com métodos computacionais.

Outra aplicação é apresentada no trabalho de \citeonline{santos2015uso}, que aplica o uso de dados de NFes no estudo de alternativas para a obtenção de dados para o planejamento do transporte de cargas urbano, como forma de levantar dados para a tomada de decisões que evitem congestionamentos ou otimizem fluxos de tráfego devido a esse tipo de transporte.

Por fim, o trabalho de \citeonline{campos2019geraccao} faz o uso de MDFes para a obtenção de matrizes Origem-Destino, uma ferramenta de visualização de demanda, mostrando em seus resultados que as matrizes obtidas através de dados de documento fiscais eram semelhantes às obtidas com métodos tradicionais.

Todos esses trabalhos atestam a riqueza, viabilidade, e confiabilidade dos dados de documentos fiscais para uso em variados domínios e aplicações.

\section{A Pandemia de COVID-19}
\label{introducao:pandemia}

O ano de 2020 foi marcado por uma pandemia de proporções globais poucas vezes vista na história da humanidade. Segundo \citeonline{zhou2020pneumonia}, a epidemia se iniciou no dia 12 de dezembro de 2019 na cidade de Wuhan, na China, causada por infecções pelo coronavírus Sars-Cov-2 capaz de causar Síndrome Respiratório Aguda Grave, doença chamada COVID-19, cujos sintomas incluem febre, tosse seca, dificuldade respiratória, dor de cabeça, podendo evoluir para casos mais graves com pneumonia, danos pulmonares, e morte.

Em 26 de fevereiro de 2020, o Brasil confirmou seu primeiro caso de infecção pelo vírus em um paciente de 61 anos vindo da Itália \cite{artigo:folha:primeiro-caso}, e sua primeira morte no dia 17 de março de 2020 na cidade de São Paulo, um homem de 62 anos \cite{artigo:folha:primeira-morte}. O preocupante espalhamento do vírus por diversos países levou a Organização Mundial da Saúde a declarar em 11 de março de 2020 à classificação de pandemia \cite{artigo:folha:oms-declara-pandemia}, após a disseminação para mais de cem países em todos os continentes.

A OMS recomenda como forma de prevenção o distanciamento social, o uso de máscaras, e a higiene contínua das mãos \cite{oms:coronavirus-disease-advice}. O alto risco de contaminação por pacientes assintomáticos e pressintomáticos infectados torna o controle da doença muito difícil, sendo os principais métodos populacionais de prevenção indicados o isolamento e a quarentena \cite{oms:coronavirus-faq}. A doença ainda não possui remédios eficazes indicados pela OMS, sendo recomendados apenas a prevenção e tratamentos intensivos para pacientes já em estado grave.

O Brasil, assim como diversos países do mundo, adotou medidas para conter o avanço do vírus. Por meio da lei nº 13.979/2020 \cite{lei:13979:medidas-para-pandemia}, alterada posteriormente pelas leis nº 14.019/2020 \cite{lei:14019:medidas-para-pandemia} e 14.035/2020 \cite{lei:14035:medidas-para-pandemia}, foram estabelecidas medidas para enfrentamento da pandemia, que no Artigo 3º prevê:

\begin{citacao}
Art. 3º  Para enfrentamento da emergência de saúde pública de importância internacional de que trata esta Lei, as autoridades poderão adotar, no âmbito de suas competências, entre outras, as seguintes medidas:

I - isolamento;

II - quarentena;

III - determinação de realização compulsória de:

a) exames médicos;
    
b) testes laboratoriais;
    
c) coleta de amostras clínicas;
    
d) vacinação e outras medidas profiláticas; ou 
    
e) tratamentos médicos específicos;
    
III-A – uso obrigatório de máscaras de proteção individual; 

...

VI – restrição excepcional e temporária, por rodovias, portos ou aeroportos, de: 

a) entrada e saída do País; e
    
b) locomoção interestadual e intermunicipal;
\end{citacao}

A adoção dessas medidas, embora não tem ocorrido de forma ordenada em todo o país, afetou a mobilidade social. Como exemplo de tais medidas, no dia 21 de março de 2020 o governo do estado de São Paulo \cite{artigo:folha:sp-fechamento-comercio} decreta o fechamento do comércio e de todos os serviços essenciais por 15 dias, medida posteriormente prolongada cujos efeitos duraram várias semanas. Também na cidade do Rio de Janeiro, foi decretado o fechamento obrigatório do comércio no dia 22 de março de 2020, também em tentativa de conter o vírus \cite{artigo:uol:rj-fechamento-comercio}. No dia 18 de maio de 2020, a cidade de São Paulo aprovou a antecipação de dois feriados \cite{artigo:folha:sao-paulo-aprova-feriados} como forma de incentivar o aumento do índice de isolamento social, que indica o percentual da população que está efetivamente respeitando o isolamento social.

A Figura~\ref{fig:intro-1.1-isolamento-social-brasil} ilustra como o índice de isolamento social variou entre fevereiro de 2020 e janeiro de 2021. O pico de isolamento ocorreu no dia 22 de março de 2020 alcançando 62.2\%, mas posteriormente diminuiu estando no patamar de 37.3\% em 15 de janeiro de 2021.

\begin{figure}[htb]
    \centering
    \caption{Gráfico do índice de isolamento social no Brasil}
    \label{fig:intro-1.1-isolamento-social-brasil}
    \includegraphics[scale=0.7]{images/intro-1.1-isolamento-social-brasil.png}
    \fdireta{inloco:mapa-isolamento-brasil}
\end{figure}

Em janeiro de 2021, a OMS registra mais de 90 milhões de casos confirmados de infecção, sendo mais de 2 milhões de mortes causadas pela doença \cite{oms:coronavirus-disease-dashboard}, sendo que a cada dia estão ainda sendo registrados em média cerca de 100 mil novos casos e 2 mil mortes. O Brasil registra mais de 8 milhões de casos confirmados sendo 200 mil mortes causadas pela doença, sendo ainda registrados em média cerca de 50 mil casos e 1 mil mortos por dia. No momento, a OMS avalia 15 vacinas candidatas a imunizantes, algumas já com pesquisas avançadas e em uso emergencial por alguns países \cite{oms:coronavirus-vaccines-status}.

O Brasil aplicou seu primeiro imunizante \cite{artigo:folha:primeira-vacina} no dia 17 de janeiro de 2021, após aprovação para uso em caráter emergencial pela \sigla{Anvisa}{Agência Nacional de Vigilância Sanitária}, em uma enfermeira de 54 anos que trabalha na UTI do Instituto de Infectologia Emílio Ribas, em São Paulo.

\section{Crise econômica}
\label{introducao:crise-economica}

O fechamento do comércio, as medidas de incentivo ao isolamento social, além dos altos números de mortos, são fatores que afetam economicamente diversos setores da economia.

\citeonline[traduç{\~a}o nossa]{mckibbin2020economic} aponta que "em um mundo fortemente conectado e integrado, os impactos da doença (COVID-19) vão além da mortalidade (aqueles que morrem) e da morbidade (aqueles que ficam incapazes de trabalhar por um período de tempo)". O estudo, datado de março de 2020, ainda previa cenários onde a pandemia poderia se manter restrita à China, mas também a cenários que hoje sabemos mais próximos da realidade com a doença se espalhando por diversos países. Este estudo já apontava a possibilidade, hoje confirmada, de um alto número de mortes e alto impacto no \sigla{PIB}{Produto Interno Bruto} nos países estudados. Mesmo nos cenários mais otimistas, de menor espalhamento do vírus, o estudo já previa um desvio negativo importante para o PIB de diversos países, podendo chegar a 9.9\% de desvio negativo nos piores casos. Segundo os dados do \sigla{IBGE}{Instituto Brasil de Geografica e Estatística}, o Brasil teve no acumulado entre o último trimestre de 2019 e os três primeiros trimestres de 2020 uma queda de 3.4\% no PIB \citeonline{intro:ibge-pib}.

O trabalho de \citeonline[traduç{\~a}o nossa]{fernandes2020economic} faz uma análise dos impactos econômicos da COVID-19 no cenário mundial ainda em março de 2020. O estudo aponta que "os impactos da doença vão além da mortalidade. Sendo que, governos por todo mundo tem se preparado para planos de contingência e pacotes de ajuda para sustentar suas economias', apontando também para o impacto dos \textit{lockdowns}, uma das medidas restritivas adotadas, na redução do consumo e em interrupções na produção. O estudo também compara o evento de 2020 com crises anteriores e descreve que tanto em relação à crise econômica de 2008 quanto em relação à pandemia de SARS de 2002 há diferenças signiticativas que tornam a crise de 2020 mais grave e abrangente.

Em outro estudo, \citeonline{sun2020did} analisa o impacto da pandemia no transporte aéreo global e como este poderia ter contribuído para a pandemia, usando dados de radares aéreos do mundo todo. Neste trabalho, \citeonline{sun2020did} mostra uma diminuição da conectividade em um grafo de conexões em aeroportos de todo o mundo se comparado o mês de janeiro com o mês de maio de 2020. Ao comparar métricas de sistemas complexos neste período, o estudo também detecta uma queda de 50\% na centralidade de grau, uma redução também da assortatividade de grau, um aumento significativo na centralidade de intermediação, e também na quantidade de comunidades da rede. Este estudo é um precedente importante para o nosso trabalho uma vez que aponta a alterações topológicas em sistemas complexos causadas pela pandemia.

Usando dados do mercado de ações, \citeonline{aslam2020network} faz uma análise da variação da correlação entre diversos mercado e mudanças estruturais do ponto de vista de sistemas complexos entre o período anterior e durante o espalhamento da COVID-19. Este estudo também identifica mudanças topológicas no relacionamento entre mercado de todo o mundo, relatando um aumento do caminho mínimo médio entre eles, além de uma diminuição na centralidade de intermediação e na centralidade de proximidade. Essa análise considera apenas a árvore geradora mínima, mas já aponta mudanças diferentes das relatadas por \citeonline{sun2020did}.

Estudos sobre o impacto econômico da crise causada pela COVID-19 ainda serão complementados por outros trabalhos, tendo em vista que ainda não alcançamos e o fim da pandemia é difícil prever quando isso acontecerá, e também por ser um evento relativamente recente e que certamente ainda será muito analisado.

\section{Objetivos}
\label{introducao:objetivos}

Tendo em vista a oportunidade de utilizar valiosos dados de documentos fiscais, o impacto econômico previsto por outros trabalhos, e a importância do evento ocorrido, este trabalho se propõe a contribuir com uma análise do impacto econômico da crise provocada pela pandemia de COVID-19 através do ponto de vista de sistemas complexos, e com uma análise do impacto de mudanças topológicas de redes na magnitude e abrangência da crise para cada entidade estudada, apresentando uma categorização de impacto econômico, e verificando se há possibilidade de utilizar esses dados para categorizá-las entre entidades afetadas e não-afetadas pela crise.

No capítulo~\ref{chapter:documenos-fiscais} será apresentado o domínio dos dados utilizados neste trabalho com uma explicação sobre documentos fiscais, a legislação brasileira envolvida, e os sistemas usados para captação dos dados. A seguir, será descrita a base de dados utilizada, com uma análise descritiva e detalhes do pré-processamento feito para a coleta dos dados no capítulo~\ref{chapter:base-de-dados}. O impacto econômico da crise é descrito no capítulo~\ref{chapter:impacto-economico} que mostra sob diferentes visões a abrangência e magnitude da mesma. No capítulo~\ref{chapter:resultados} é feita uma análise descritiva com diferentes modelagens e métricas das redes utilizadas e são mostradas possíveis alterações estruturais no ponto de vista de sistemas complexos, com diferentes modelagens, e uma tentativa de identificação de características que podem levar a categorizar um agente econômico como afetado ou não pela pandemia. O capítulo~\ref{chapter:conclusão} encerra este trabalho com as conclusões obtidas e os próximos passos propostos por este trabalho.


\chapter{Documentos Fiscais}
\label{chapter:documenos-fiscais}

A legislação fiscal brasileira é notoriamente conhecida por ser complexa. Nas duas últimas décadas, representantes das três esferas do governo brasileiro, municipal, estadual, e federal, se reuniram para tentar mudar essa realidade promovendo uma atuação mais integrada e modernizando o sistema. O mercado brasileiro também respondeu a essas mudanças criando oportunidades que visam facilitar o cumprimento das obrigações fiscais de empresas de todo país.

\section{Legislação Brasileira}
\label{section:documentos-fiscais:legislacao}

O Sistema Tributário Nacional foi instituído pela lei nº 5172 de 1966 \cite{lei:5172:codigo-tributario}, há mais de cinco décadas. Desde então passou por diversas reformas visando aumentar sua eficiência, modernizando-o. O artigo 37 da Constituição Federal, alterado pela Emenda Constitucional nº 42 de 19 de dezembro de 2003 \cite{constituicao:emenda42_2003}, em seu vigésimo segundo inciso prevê:

\begin{citacao}
Art. 37. A administração pública direta e indireta de qualquer dos Poderes da União, dos Estados, do Distrito Federal e dos Municípios obedecerá aos princípios de legalidade, impessoalidade, moralidade, publicidade e eficiência e, também, ao seguinte:

...

XXII - as administrações tributárias da União, dos Estados, do Distrito Federal e dos Municípios, atividades essenciais ao funcionamento do Estado, exercidas por servidores de carreiras específicas, terão recursos prioritários para a realização de suas atividades e atuarão de forma integrada, inclusive com o compartilhamento de cadastros e de informações fiscais, na forma da lei ou convênio \cite{constituicao:1988}.
\end{citacao}

O compartilhamento de cadastros e de informações previsto neste artigo permitiu então que as esferas do governo brasileiro tomassem atitudes no sentido de centralizar sistemas tributários, criando protocolos e sistemas para tal.

Desde então, diversos secretários da Fazenda Federal, das Fazendas Estaduais e do Distrito Federal, e de Fazendas Municipais se reúnem anualmente no \sigla{ENAT}{Encontro Nacional de Administradores Tributários} para discutir como modernizar e tornar mais efetivo o Sistema Tributário Brasileiro. A seguir, são descritos alguns importantes protocolos assinados nesses encontros.

\subsection{ENAT}

Em julho de 2004, em Salvador-BA, ocorre a primeira edição do ENAT. Neste encontro, foram assinados dois protocolos de cooperação  técnica para a instalação do Projeto Cadastro Sincronizado e do Projeto de Escrituração Digital.

No protocolo de Projeto Cadastro Sincronizado, fica previsto a construção de um cadastro de contribuintes sincronizado seguindo as seguintes diretivas:

\begin{citacao}
Na construção do cadastro referido na cláusula primeira, serão observados os seguintes parâmetros, entre outros que vierem a ser definidos de comum acordo pelos partícipes:

I - entrada de dados única;

II - bases de dados independentes, porém sincronizadas;

III - reciprocidade na aceitação da legislação de cada ente signatário;

IV - adoção do número de inscrição no Cadastro Nacional da Pessoa Jurídica (CNPJ) \sigla*{CNPJ}{Cadastro Nacional da Pessoa Jurídica} como identificador cadastral dos contribuintes do ICMS \sigla*{ICMS}{Imposto Sobre Operações Relativas à Circulação de Mercadorias e Prestações de Serviços de Transporte Interestadual e Intermunicipal e de Comunicação} e ISS \sigla*{ISS}{Imposto Sobre Serviços de Qualquer Natureza} \cite{enat:2004:protocolo1}.
\end{citacao}

Já no protocolo de Projeto de Escrituração Digital, são criadas iniciativas para a modernização da administração tributária brasileira segundo os seguintes parâmetros:

\begin{citacao}
A administração tributária está assentada sobre três pilares básicos, a saber, o cadastro de contribuintes, que permita a perfeita identificação e individualização das pessoas, o documento básico de comércio, Nota Fiscal, que registre a atividade comercial com suas particularidades e a codificação das mercadorias e serviços \cite{enat:2004:protocolo1};
\end{citacao}

Mais adiante, o documento define as seguintes prioridades a serem atacadas, a destacar:

\begin{citacao}
IV – criar grupo de trabalho para elaborar proposta com vistas à adaptação da codificação
da Nomenclatura Comum do Mercosul - NCM \sigla*{NCM}{Nomenclatura Comum do Mercosul} às especificidades tributárias do ICMS;

IX - investir no desenvolvimento de modelo de dados único e padronizado para todos os
Fiscos, relativamente às demais ferramentas de Administração Tributárias;

X - investir na regulamentação do uso da certificação digital em todos os documentos
fiscais, utilizado o programa de Transmissão Eletrônica de Documentos - TED \sigla*{TED}{Transmissão Eletrônica de Documentos}, como padrão
nacional;

XI - investir na harmonização da legislação das Unidades da Federação e, na
padronização da escrituração fiscal e das informações econômico-fiscais;

XII – compartilhar os sistemas de auditoria fiscal existentes nas diversas Unidades
Federadas e promover o desenvolvimento conjunto de novas ferramentas de apoio a ação fiscal;

XIII – desenvolver e disponibilizar aos contribuintes programa que permita a captura e
transmissão on-line dos documentos fiscais emitidos para as administrações fiscais; 
\end{citacao}

Em agosto do ano seguinte, em São Paulo-SP, ocorre a segunda edição do ENAT, adicionando quatro importantes protocolos de cooperação técnica, das quais ressaltam-se três.

O segundo estabelece o \sigla{SPED}{Sistema Público de Escrituração Digital} sob os seguintes pressupostos:

\begin{citacao}
I - bases de dados compartilhadas entre as Administrações Tributárias;

II - reciprocidade na aceitação da legislação de cada ente signatário, relativa aos livros
contábeis e fiscais;

III - validade jurídica dos livros contábeis e fiscais em meio digital, dispensando a
emissão e guarda de documentos e livros em papel;

IV - eliminação da redundância de informações através da padronização e racionalização
das obrigações acessórias;

V - preservação do sigilo fiscal, nos termos do Código Tributário Nacional \cite{enat:2005:protocolo2}. 
\end{citacao}

O terceiro estabelece a Nota Fiscal Eletrônica, parte integrante do SPED, definindo:

\begin{citacao}
I - substituição das notas fiscais em papel por documento eletrônico;

II - validade jurídica dos documentos digitais;

III - padronização nacional da NF-e;

IV - mínima interferência no ambiente operacional do contribuinte;

V - compartilhamento da NF-e entre as administrações tributárias;

VI - preservação do sigilo fiscal, nos termos do Código Tributário Nacional \cite{enat:2005:protocolo3}. 
\end{citacao}

E por fim, a padronização e aplicação da \sigla{CNAE}{Classificação Nacional de Atividades Econômicas} para o Cadastro Sincronizado de Contribuintes \cite{enat:2005:protocolo4}.

Essas iniciativas viriam possibilitar então a criação de sistemas automatizados de controle fiscal. Em especial, a criação do SPED e da NF-e possibilitaram o uso de documentos fiscais não só com a finalidade de simplificar as obrigações fiscais das empresas brasileiras, mas também sua utilização como artefato de controle interno, devido à sua riqueza de informações estruturadas.

Na edição seguinte, em 2006, o ENAT ocorreu em Fortaleza-CE e foram estabelecidos protocolos de cooperação técnica que seguiram adicionando novos artefatos, os quais ressaltam-se a implantação da \sigla{NFSe}{Nota Fiscal de Serviço Eletrônica} e do Conhecimento de Transporte Eletrônico, outros dois documentos fiscais eletrônicos. O encontro segue ocorrendo anualmente e os protocolos tem se aperfeiçoado para melhorar e integrar o sistema tributário e fiscal brasileiro.

\subsection{Documentos Fiscais Eletrônicos}
\label{section:documentos-fiscais:dfe}

Existem diversos documentos fiscais, como já mencionado neste trabalho. As empresas tem a obrigação fiscal de emitir e armazenar estes documentos. O prazo prescricionário destes documentos é apresentado nos artigos 173 e 174 do Código Tributário Nacional \cite{lei:5172:codigo-tributario}:

\begin{citacao}
Art. 173. O direito de a Fazenda Pública constituir o crédito tributário extingue-se após 5 (cinco) anos, contados:

I - do primeiro dia do exercício seguinte àquele em que o lançamento poderia ter sido efetuado;

II - da data em que se tornar definitiva a decisão que houver anulado, por vício formal, o lançamento anteriormente efetuado.

Parágrafo único. O direito a que se refere este artigo extingue-se definitivamente com o decurso do prazo nele previsto, contado da data em que tenha sido iniciada a constituição do crédito tributário pela notificação, ao sujeito passivo, de qualquer medida preparatória indispensável ao lançamento.

Art. 174. A ação para a cobrança do crédito tributário prescreve em cinco anos, contados da data da sua constituição definitiva.

Parágrafo único. A prescrição se interrompe:

I – pelo despacho do juiz que ordenar a citação em execução fiscal;

II - pelo protesto judicial;

III - por qualquer ato judicial que constitua em mora o devedor;

IV - por qualquer ato inequívoco ainda que extrajudicial, que importe em reconhecimento do débito pelo devedor.
\end{citacao}

Ou seja, as empresas tem a obrigação da guarda de documentos por pelo menos cinco anos, sendo maior a depender do tipo de documento fiscal.

A \sigla{Sefaz}{Secretaria de Estado da Fazenda}, um órgão vinculado ao Ministério da Fazenda, disponibiliza o Portal da Nota Fiscal Eletrônica \cite{portal-sefaz}, onde as empresas brasileiras podem consultar os seus documentos fiscais utilizando seu certificado digital. Também são disponibilizados serviços \textit{web} para a emissão e consulta dos documentos fiscais \cite{portal-sefaz-webservices}, vinculados às Secretarias da Fazenda estaduais. Algumas unidades federativas utilizam o Ambiente Nacional, em uma proposta de unificação dos sistemas.

Podemos dizer que a Nota Fiscal Eletrônica é o principal documento dentre os documentos fiscais, uma vez que é ela quem efetivamente descreve todos as movimentações de mercadorias de uma empresa. A Nota Fiscal Eletrônica é disponibilizada no formato \sigla{XML}{Extensible Markup Language} \cite{portal-w3c-xml}, uma linguagem de marcação comum em protocolos SOAP. Um exemplo da estrutura e conteúdo de uma NFe está disponível no Apêndice~\ref{chapter:nfe}

\begin{figure}[htb]
    \centering
    \caption{Exemplo de DANFE através do Portal Arquivei}
    \label{fig:nfe}
    \includegraphics[scale=0.25]{images/nfe.png}
    \fautor
\end{figure}

A disponibilização dos documentos é feita conforme um critério determinado pela Sefaz em que a empresa tem acesso ao documento de acordo com o papel que ela exerce nele. Por exemplo, NFes são disponibilizadas ao destinatário de uma NFe mas não ao emitente. Desta forma, para obter a completude de todos os documentos fiscais de uma empresa de forma automatizada, é necessário que o sistema emitente também armazene os dados.

\section{A Arquivei}
\label{section:documentos-fiscais:arquivei}

Os dados obtidos neste trabalho foram adquiridos através da empresa parceira, Arquivei \cite{portal-arquivei}. A Arquivei é uma empresa especializada em inteligência em documentos fiscais, fundada em 2013 na cidade São Carlos no interior de São Paulo. A empresa disponibiliza um sistema automatizado de consulta de documentos fiscais, utilizando os certificados digitais de seus clientes para obter e armazenar os dados de documentos fiscais. Através do portal, os clientes podem então não apenas cumprir com sua obrigação tributária, mas emitir relatórios e automatizar processos internos das empresas a partir desses dados.

Através dos dados dos clientes da empresa, é possível obter dados de terceiros por meio de um efeito de rede, uma vez que documentos fiscais podem citar diversas empresas. Esse efeito de rede é importante pois aumenta a abrangência dos dados analisados.

A empresa tem hoje mais de 150 funcionários, 12 mil clientes, e atende mais de 85 mil empresas de todo o Brasil. Este trabalho utiliza uma amostra da base de dados da empresa, cuja abrangência será descrita em detalhes no capítulo~\ref{chapter:base-de-dados}.

\subsection{Segurança dos dados}
\label{section:documentos-fiscais:seguranca}

O tratamento dos dados nos ambientes da empresa parceira foram executados apenas pelo autor deste trabalho, de maneira anonimizada, mantendo o sigilo e privacidade previstos na legislação atual e os contratos de prestação de serviço entre a empresa parceira e seus clientes, além de cumprir os critérios previstos no contrato de trabalho entre a empresa parceira e autor deste trabalho. Os dados são armazenados pela empresa parceira utilizando avançados critérios e procedimentos de segurança e com acesso restrito, não tendo tais critérios e procedimentos sido comprometidos de forma alguma na execução deste trabalho.


\chapter{Redes Complexas}
\label{chapter:redes}

\section{Conceitos}

Newman \cite{newman:2018:networks} define uma rede da seguinte forma: \aspas{Uma \textbf{rede} é, em sua forma mais simples, uma coleção de pontos unidos em pares por linhas. Na nomenclatura da área um ponto é chamado \textbf{nó} ou \textbf{vértice} e a linha é chamada \textbf{aresta}}.  Uma rede é também chamada de \textbf{grafo} na literatura matemática.

A quantidade de vértices de um grafo $G$ é denotado por $n$ e a quantidade de arestas é denotado por $m$. Uma aresta que conecta um vértice a ele mesmo é chamado \textbf{ciclo}. Um grafo pode ter mais de uma aresta entre dois vértices, neste caso estas são chamadas \textbf{multiarestas}, e o grafo é chamado \textbf{multigrafo}. Um grafo que não possui ciclos é chamado \textbf{grafo acíclico}. Um grafo que não possui ciclos nem multiarestas é chamado \textbf{grafo simples}.

As arestas de um grafo podem ou não possuir uma direção associada, indicando que cada aresta parte de um vértice em direção a outro. Um grafo que possui arestas direcionadas é chamado \textbf{grafo direcionado}, também chamado \textbf{dígrafo}, caso contrário é chamado \textbf{grafo não-direcionado}. As arestas de um grafo também podem ou não possuir valores associados, chamados de \textbf{pesos}. Um grafo que possui pesos é chamado \textbf{grafo ponderado}.

\section{Representação}

Um grafo é comumente representado por uma matriz de adjacências na forma:

\begin{equation}
A_{ij} = \begin{cases}
1 & \mbox{se existe uma aresta entre os vértices i e j,}\\
0 & \mbox{caso contrário}
\end{cases}
\end{equation}

No caso de um grafo ponderado, a matriz toma a seguinte forma:

\begin{equation}
A_{ij} = \begin{cases}
w_{ij} & \mbox{se existe uma aresta entre os vértices i e j de peso $w_{ij}$,}\\
0 & \mbox{caso contrário}
\end{cases}
\end{equation}

\begin{figure}[htb]
 \caption{Exemplo de grafo acíclico direcionado não-ponderado}
 \label{fig:redes1:grafo-exemplo}
 \centering
 \includegraphics[scale=0.7]{images/redes-1-grafo-exemplo.png}
 \fautor
\end{figure}

A figura \ref{fig:redes1:grafo-exemplo} representa um grafo que pode ser representado pela matriz:

\begin{equation}
A = \begin{bmatrix}
0 & 1 & 0 & 1 & 1 \\
0 & 0 & 1 & 1 & 0 \\
0 & 0 & 0 & 1 & 0 \\
1 & 0 & 0 & 0 & 0 \\
1 & 0 & 1 & 0 & 0 \\
\end{bmatrix}
\end{equation}

outros conceitos usados na seção seguinte

\section{Métricas}

descricao das métricas usadas


\chapter{Base de Dados}
\label{chapter:base-de-dados}

Neste capítulo, serão descritos os conjuntos de dados utilizados neste trabalho, bem como os processos de extração e transformação envolvidos.

\section{Dados de Documentos Fiscais}

Nesta seção, será apresentado o tratamento dos dados obtidos através da coleta de documentos fiscais.

\subsection{Aquisição}

Como descrito na Seção~\ref{section:arquivei}, os dados utilizados neste trabalho são obtidos através do armazém de dados da empresa parceira. Estes dados passam por um complexo processo de ingestão e tratamento descrito em detalhes no Apêndice~\ref{chapter:armazem-de-dados}.

A seleção dos dados foi feita usando o armazém de dados, através da linguagem \sigla{SQL}{\textit{Structured Query Language}}. Os critérios de seleção utilizados para obter os dados do armazém de dados foram:

\begin{enumerate}
    \item Foram utilizados apenas dados de Notas Fiscais Eletrônicas, excluindo outros documentos fiscais
    \item Foram utilizados apenas documentos obtidos através do \textit{webservice} da Sefaz
    \item Foram utilizados apenas dados de documentos emitidos entre 01 de janeiro de 2019 e 31 de outubro de 2020
    \item Foram utilizados apenas documentos contendo a assinatura digital do emitente
    \item Foram utilizados apenas documentos contendo o protocolo de processamento do \textit{webservice} da Sefaz
    \item Foram utilizados apenas documentos cuja emissão foi autorizada pelo \textit{webservice} da Sefaz
    \item Foram utilizados apenas documentos referentes a transações cujo \sigla{CFOP}{Código Fiscal de Operações e Prestações} se refere às movimentações entre empresas envolvendo comercialização de mercadorias. Esses critérios são explicados em detalhes no Apêndice~\ref{chapter:cfop}.
\end{enumerate}

Chamamos de \textbf{empresas recorrentes} aquelas que possuem pelo menos um documento na base de dados emitido para cada mês entre janeiro de 2019 e outubro de 2020. Para remover eventuais problemas de completude nos dados, foram considerados apenas documentos de transações entre empresas recorrentes.

Todos os dados obtidos a partir do armazém de dados foram anonimizados. Dados identificadores ou sensíveis foram removidos e reindexados de forma a manter o sigilo e privacidade de tais dados respeitando tanto a legislação atual quanto os contratos de prestação de serviços da empresa parceira.

\subsection{Pré-processamento}

Esses dados foram tratados

dados dividos em 4 tabelas

A base de dados de documentos fiscais utilizada neste trabalho considerou um subconjunto dos dados 

\section{Dados Abertos de Empresas Brasileiras}

Nesta seção, será apresentado o tratamento dos dados obtidos através da base de dados abertos da Receita Federal.

\subsection{Aquisição}

\sigla*{UF}{Unidade Federativa}
\sigla*{Sefaz}{Secretaria da Fazenda}

\section{Análise Descritiva da Base de Dados}

Nesta seção, será apresentada uma análise descritiva dos dados utilizados neste trabalho

\subsection{Abrangência territorial}

\begin{figure}[htb]
    \centering
    \caption{Percentual das empresas de cada região brasileira presente na base de dados}
    \includegraphics[scale=0.7]{images/base-de-dados-1.1-presenca-por-regiao.png}
    \label{fig:base-de-dados:descritiva-1.1-presenca-por-regiao-1.1}
    \fautor
\end{figure}

\begin{figure}[htb]
    \centering
    \caption{Quantidade de empresas de cada região brasileira presente na base de dados relativa ao total}
    \includegraphics[scale=0.7]{images/base-de-dados-1.2-qtde-por-regiao.png}
    \label{fig:base-de-dados:descritiva-1.2-qtde-por-regiao}
    \fautor
\end{figure}

\begin{figure}[htb]
    \centering
    \caption{Percentual das empresas das cada UF presente na base de dados}
    \includegraphics[scale=0.7]{images/base-de-dados-2.1-presenca-por-uf.png}
    \label{fig:base-de-dados:descritiva-2.1-presenca-por-uf}
    \fautor
\end{figure}

\begin{figure}[htb]
    \centering
    \caption{Quantidade de empresas de cada UF presente na base de dados relativa ao total}
    \includegraphics[scale=0.7]{images/base-de-dados-2.2-qtde-por-uf.png}
    \label{fig:base-de-dados:descritiva-2.2-qtde-por-uf}
    \fautor
\end{figure}

\begin{figure}[htb]
    \centering
    \caption{Histograma do percentual das empresas de cada município brasileiro presente na base de dados}
    \includegraphics[scale=0.7]{images/base-de-dados-3.1-presenca-por-mun.png}
    \label{fig:base-de-dados:descritiva-3.1-presenca-por-mun}
    \fautor
\end{figure}

\subsection{Abrangência territorial}

\begin{figure}[htb]
    \centering
    \caption{Percentual das empresas presente na base de dados categorizadas por porte}
    \includegraphics[scale=0.7]{images/base-de-dados-4.1-presenca-por-porte.png}
    \label{fig:base-de-dados:descritiva-4.1-presenca-por-porte}
    \fautor
\end{figure}

\begin{figure}[htb]
    \centering
    \caption{Quantidade de empresas categorizadas por porte presente na base de dados relativa ao total}
    \includegraphics[scale=0.7]{images/base-de-dados-4.2-qtde-por-porte.png}
    \label{fig:base-de-dados:descritiva-4.2-qtde-por-porte}
    \fautor
\end{figure}

\begin{figure}[htb]
    \centering
    \caption{Percentual das empresas presente na base de dados categorizadas pela opção por MEI}
    \includegraphics[scale=0.7]{images/base-de-dados-5.1-presenca-por-mei.png}
    \label{fig:base-de-dados:descritiva-5.1-presenca-por-mei}
    \fautor
\end{figure}

\begin{figure}[htb]
    \centering
    \caption{Quantidade de empresas categorizadas pela opção por MEI presente na base de dados relativa ao total}
    \includegraphics[scale=0.7]{images/base-de-dados-5.2-qtde-por-mei.png}
    \label{fig:base-de-dados:descritiva-5.2-qtde-por-mei}
    \fautor
\end{figure}

\begin{figure}[htb]
    \centering
    \caption{Percentual das empresas presente na base de dados categorizadas pela opção pelo Simples Nacional}
    \includegraphics[scale=0.7]{images/base-de-dados-6.1-presenca-por-simples-nacional.png}
    \label{fig:base-de-dados:descritiva-6.1-presenca-por-porte}
    \fautor
\end{figure}

\begin{figure}[htb]
    \centering
    \caption{Quantidade de empresas categorizadas pela opção pelo Simples Nacional presente na base de dados relativa ao total}
    \includegraphics[scale=0.7]{images/base-de-dados-6.2-qtde-por-simples-nacional.png}
    \label{fig:base-de-dados:descritiva-6.2-qtde-por-porte}
    \fautor
\end{figure}

\begin{figure}[htb]
	\centering
    \caption{Histograma do percentual das empresas categorizadas por idade presente na base de dados}
    \includegraphics[scale=0.7]{images/base-de-dados-10.1-presenca-por-idade.png}
    \label{fig:base-de-dados:descritiva-10.1-presenca-por-idade}
    \fautor
\end{figure}

\begin{figure}[htb]
	\centering
    \caption{Histograma da quantidade de empresas categorizadas por idade presente na base de dados}
    \includegraphics[scale=0.7]{images/base-de-dados-10.2-qtde-por-idade.png}
    \label{fig:base-de-dados:descritiva-10.2-qtde-por-idade}
    \fautor
\end{figure}

\subsection{Abrangência setorial}

\begin{figure}[htb]
    \centering
    \caption{Percentual das empresas presente na base de dados categorizadas pela seção do CNAE}
    \includegraphics[scale=0.7]{images/base-de-dados-7.1-presenca-por-secao.png}
    \label{fig:base-de-dados:descritiva-7.1-presenca-por-secao}
    \fautor
\end{figure}

\begin{figure}[htb]
    \centering
    \caption{Quantidade de empresas categorizadas seção do CNAE presente na base de dados relativa ao total}
    \includegraphics[scale=0.7]{images/base-de-dados-7.2-qtde-por-secao.png}
    \label{fig:base-de-dados:descritiva-7.2-qtde-por-secao}
    \fautor
\end{figure}

\begin{figure}[htb]
    \centering
    \caption{Percentual das empresas presente na base de dados categorizadas pela divisão do CNAE}
    \label{fig:base-de-dados:descritiva-8.1-presenca-por-divisao} 
    \begin{subfigure}[b]{1.0\textwidth} 
        \includegraphics[scale=0.7]{images/base-de-dados-8.1.1-presenca-por-divisao.png}
        \caption{Divisão 01 a 47}
        \label{fig:base-de-dados:descritiva-8.1.1-presenca-por-divisao}
    \end{subfigure} ~ \\
    \begin{subfigure}[b]{1.0\textwidth}
        \includegraphics[scale=0.7]{images/base-de-dados-8.1.2-presenca-por-divisao.png}
        \caption{Divisão 49 a 99}
        \label{fig:base-de-dados:descritiva-8.1.2-presenca-por-divisao}
    \end{subfigure}
    \fautor
\end{figure}

\begin{figure}[htb] 
    \centering 
    \caption{Percentual das empresas presente na base de dados categorizadas pela divisão do CNAE}
    \label{fig:base-de-dados:descritiva-8.2-qtde-por-divisao} 
    \begin{subfigure}[b]{1.0\textwidth}
        \includegraphics[scale=0.7]{images/base-de-dados-8.2.1-qtde-por-divisao.png}
        \caption{Divisão 01 a 47}
        \label{fig:base-de-dados:descritiva-8.2.1-qtde-por-divisao}
    \end{subfigure} ~ \\
    \begin{subfigure}[b]{1.0\textwidth} 
        \includegraphics[scale=0.7]{images/base-de-dados-8.2.2-qtde-por-divisao.png}
        \caption{Divisão 49 a 99}
        \label{fig:base-de-dados:descritiva-8.2.2-qtde-por-divisao}
    \end{subfigure}
    \fautor
\end{figure}

\begin{figure}[htb]
	\centering
    \caption{Histograma do percentual das empresas de cada CNAE presente na base de dados}
    \includegraphics[scale=0.7]{images/base-de-dados-9.1-presenca-por-cnae.png}
    \label{fig:base-de-dados:descritiva-9.1-presenca-por-cnae}
    \fautor
\end{figure}


\chapter{Impacto Econômico da Pandemia}
\label{chapter:impacto-economico}

Neste seção, descrevemos o possível impacto econômico da Pandemia de COVID-19, contextualizada no Capítulo~\ref{chapter:pandemia}, verificando mensalmente se o volume transacionado foi próximo ou não fo volume esperado.

\section{Cenário geral}

Nessa seção, descrevemos os dados da base totalizados mensalmente de forma agregada.

\begin{figure}[htb]
	\centering
    \caption{Valor total transacionado por mês no período analisado}
    \includegraphics[scale=0.7]{images/base-de-dados-18.1-valor-mensal-total.png}
    \label{fig:pandemia:descritiva-18.1-valor-mensal-total}
    \fdadospesquisa
\end{figure}

\begin{figure}[htb]
	\centering
    \caption{Comparação do valor total transacionado por mês entre 2019 e 2020 (Parte 4)}
    \includegraphics[scale=0.7]{images/base-de-dados-19.1-comparacao-valor-mensal-total.png}
    \label{fig:pandemia:descritiva-19.1-comparacao-valor-mensal-total}
    \fdadospesquisa
\end{figure}

A Figura~\ref{fig:pandemia:descritiva-18.1-valor-mensal-total} mostra o valor total transacionado em cada mês entre janeiro de 2019 e outubro de 2020, onde é visível um menor valor transacionado principalmente em abril e maio de 2020.

\begin{table}[htb]
\centering
\caption{Variação entre o valor transacionado para cada mês em 2020 em relação a 2019}
\label{tab:comparacao-valor-mensal-total}
\begin{tabular}{lr}
\toprule
Mês & Variação \\
\midrule
Janeiro &   25.49\% \\
Fevereiro & 14.58\% \\
Março &     12.71\% \\
Abril &    -12.24\% \\
Maio &     -11.24\% \\
Junho &      9.20\% \\
Julho &     11.64\% \\
Agosto &     7.52\% \\
Setembro &  17.02\% \\
Outubro &    3.33\% \\
\bottomrule
\end{tabular}
\fdadospesquisa
\end{table}

A Figura~\ref{fig:pandemia:descritiva-19.1-comparacao-valor-mensal-total} mostra uma comparação dos valores mensais transacionados em 2020 em relação aos valores mensais de 2019. Verifica-se então que os meses de abril e maio tiveram um valor total transacionado menor no ano de 2020, cuja variação está descritos em detalhes na Tabela~\ref{tab:comparacao-valor-mensal-total}.

Como descrito no Capítulo~\ref{chapter:pandemia}, os meses de março e abril marcam exatamente o início do isolamento e das quarentenas no Brasil, e a aparentemente o impacto econômico nesses momentos foi mais importante.

Aplicando uma regressão linear a partir dos dados entre janeiro de 2019 e fevereiro de 2020 é possível obter uma da previsão de valor transacionado para os meses seguintes e então comparar com o valor obtido nos meses seguintes: A Figura~\ref{fig:pandemia:descritiva-18.2-valor-total-vs-previsao} mostra os valores obtidos em relação aos valores previstos, detalhados na Tabela~\ref{tab:pandemia:valor-total-vs-previsao}.

A partir desses dados, é possível verificar que no período entre março e junho de 2020 o valor transacionado total foi menor que o previsto, verificando uma retomada posterior.

\begin{figure}[htb]
	\centering
    \caption{Valor total transacionado por mês comparado com a previsão para o período analisado}
    \includegraphics[scale=0.7]{images/base-de-dados-18.2-valor-total-vs-previsao.png}
    \label{fig:pandemia:descritiva-18.2-valor-total-vs-previsao}
    \fdadospesquisa
\end{figure}

\begin{table}[htb]
\centering
\caption{Variação entre o valor transacionado para cada mês em relação ao previsto após o início da pandemia}
\label{tab:pandemia:valor-total-vs-previsao}
\begin{tabular}{lr}
\toprule
Mês & Variação \\
\midrule
Março &     -4.46\% \\
Abril &    -36.22\% \\
Maio &     -23.11\% \\
Junho &     -9.15\% \\
Julho &      8.28\% \\
Agosto &     2.09\% \\
Setembro &   4.18\% \\
Outubro &    3.46\% \\ \hline
\textbf{Total} & -6.58\% \\
\bottomrule
\end{tabular}
\fdadospesquisa
\end{table}

\section{Cenário regional}

\begin{figure}[htb]
	\centering
    \caption{Valor total transacionado por região no período analisado}
    \includegraphics[scale=0.7]{images/base-de-dados-11.1-valor-total-por-regiao.png}
    \label{fig:pandemia:descritiva-11.1-valor-total-por-regiao}
    \fdadospesquisa
\end{figure}

\begin{table}[htb]
\centering
\caption{Diferença entre valores transacionados previstos e obtidos por região no período da pandemia}
\label{tab:pandemia:variacao-por-regiao}
\begin{subtable}[h]{\textwidth}
    \label{tab:pandemia:variacao-mensal-por-regiao}
    \centering
    \begin{tabular}{c|r|r|r|r|r|r|r|r}
        \toprule
        \textbf{Região} & Março & Abril & Maio & Junho & Julho & Agosto & Setembro & Outubro \\
        \midrule
        \textbf{1} & -13.4\% & -48.8\% & -26.1\% &  -9.9\% &  8.9\% &  1.6\% &  0.7\% &  6.4\% \\
        \textbf{2} &  -9.2\% & -39.6\% & -22.3\% &  -5.3\% & 15.2\% & 10.1\% & 12.9\% & 18.9\% \\
        \textbf{3} &   8.2\% & -27.7\% & -10.7\% &   3.5\% &  9.6\% &  6.5\% &  8.4\% &  1.9\% \\
        \textbf{4} &  -3.0\% & -34.9\% & -23.8\% & -10.2\% &  3.0\% &  1.5\% &  3.4\% &  2.5\% \\
        \textbf{5} &  -8.5\% & -37.3\% & -24.9\% & -12.6\% & 22.5\% & -2.3\% &  1.6\% & -2.9\% \\
        \bottomrule
    \end{tabular}
    \caption{Diferença entre valor mensal previsto e obtido}
\end{subtable} ~ \\
\begin{subtable}[h]{0.45\textwidth}
    \label{tab:pandemia:variacao-total-por-regiao}
    \centering
    \begin{tabular}{l|r}
        \toprule
        Região & Variação total \\
        \midrule
        \textbf{1} & -9.4\% \\
        \textbf{2} & -1.9\% \\
        \textbf{3} &  0.1\% \\
        \textbf{4} & -7.5\% \\
        \textbf{5} & -7.7\% \\
        \bottomrule
    \end{tabular}
    \caption{Diferença entre valor total previsto e obtido}
\end{subtable}
\fdadospesquisa
\end{table}


\begin{figure}[htb] 
    \centering 
    \caption{Comparação do valor mensal transacionado por região entre 2019 e 2020}
    \label{fig:pandemia:descritiva-13-comparacao-valor-total-por-regiao} 
    \begin{subfigure}[b]{0.45\textwidth}
        \includegraphics[scale=0.45]{images/base-de-dados-13.1-comparacao-valor-total-por-regiao.png}
        \caption{Região 1}
        \label{fig:pandemia:descritiva-13.1-comparacao-valor-total-por-regiao}
    \end{subfigure} ~ \quad
    \begin{subfigure}[b]{0.45\textwidth}
        \includegraphics[scale=0.45]{images/base-de-dados-13.2-comparacao-valor-total-por-regiao.png}
        \caption{Região 2}
        \label{fig:pandemia:descritiva-13.2-comparacao-valor-total-por-regiao}
    \end{subfigure} ~ \\
    \begin{subfigure}[b]{0.45\textwidth}
        \includegraphics[scale=0.45]{images/base-de-dados-13.3-comparacao-valor-total-por-regiao.png}
        \caption{Região 3}
        \label{fig:pandemia:descritiva-13.3-comparacao-valor-total-por-regiao}
    \end{subfigure} ~ \quad
    \begin{subfigure}[b]{0.45\textwidth}
        \includegraphics[scale=0.45]{images/base-de-dados-13.4-comparacao-valor-total-por-regiao.png}
        \caption{Região 4}
        \label{fig:pandemia:descritiva-13.4-comparacao-valor-total-por-regiao}
    \end{subfigure} ~ \\
    \begin{subfigure}[b]{0.45\textwidth} 
        \includegraphics[scale=0.45]{images/base-de-dados-13.5-comparacao-valor-total-por-regiao.png}
        \caption{Região 5}
        \label{fig:pandemia:descritiva-13.5-comparacao-valor-total-por-regiao}
    \end{subfigure}
    \fdadospesquisa
\end{figure}

\begin{figure}[htb]
	\centering
    \caption{Valor total transacionado por UF no período analisado}
    \includegraphics[scale=0.7]{images/base-de-dados-14.1-valor-total-por-uf.png}
    \label{fig:pandemia:descritiva-14.1-valor-total-por-uf}
    \fdadospesquisa
\end{figure}

\begin{table}[htb]
\centering
\caption{Diferença entre valores mensais transacionados previstos e obtidos por UF no período da pandemia}
\label{tab:pandemia:variacao-mensal-por-uf}
    \begin{tabular}{c|r|r|r|r|r|r|r|r}
        \toprule
        \textbf{UF} & Março & Abril & Maio & Junho & Julho & Agosto & Setembro & Outubro \\
        \midrule
        \textbf{AC} &  -7.3\% & -33.9\% &  -5.1\% &   9.2\% &   27.3\% &    20.5\% &    20.6\% &   15.4\% \\
        \textbf{AL} & -11.0\% & -33.6\% & -26.5\% & -12.3\% &    6.9\% &     3.9\% &    15.4\% &    9.0\% \\
        \textbf{AM} & -15.8\% & -62.5\% & -31.5\% & -14.3\% &    0.5\% &    -4.5\% &    -8.4\% &    4.1\% \\
        \textbf{AP} & -15.6\% & -29.1\% & -17.5\% &  -8.1\% &   17.4\% &    23.9\% &    22.1\% &   12.9\% \\
        \textbf{BA} &  -0.4\% & -36.0\% & -21.7\% &  -3.3\% &   11.0\% &     4.9\% &    12.7\% &   13.5\% \\
        \textbf{CE} & -15.6\% & -45.1\% & -26.3\% &  -7.7\% &   10.8\% &     7.6\% &     7.4\% &   19.1\% \\
        \textbf{DF} &  32.3\% & -18.6\% & -19.0\% &  16.4\% &    9.8\% &     2.8\% &    10.7\% &    6.6\% \\
        \textbf{ES} &  -6.5\% & -40.1\% & -24.2\% &  -5.8\% &    6.1\% &     5.6\% &    16.2\% &    6.4\% \\
        \textbf{GO} &  10.8\% & -29.5\% &  -7.9\% &   4.0\% &    9.5\% &     1.3\% &     7.3\% &    0.4\% \\
        \textbf{MA} & -20.9\% & -31.2\% & -24.8\% &   0.2\% &   22.3\% &    21.2\% &     6.8\% &   13.7\% \\
        \textbf{MG} &  -9.0\% & -35.1\% & -20.6\% &  -6.7\% &   12.2\% &     9.1\% &    12.0\% &   11.1\% \\
        \textbf{MS} &   3.4\% & -16.0\% &  -9.1\% &  -1.6\% &   12.8\% &     5.8\% &    -3.5\% &  -15.3\% \\
        \textbf{MT} & -13.3\% & -38.9\% & -13.4\% &  -3.2\% &    7.0\% &    24.5\% &    19.8\% &   17.0\% \\
        \textbf{PA} & -13.6\% & -33.3\% & -25.8\% &  -8.8\% &   18.5\% &     5.9\% &     9.7\% &    6.1\% \\
        \textbf{PB} &  -8.7\% & -38.5\% & -11.6\% &   1.7\% &   36.1\% &    15.8\% &    20.2\% &   30.7\% \\
        \textbf{PE} & -14.6\% & -43.4\% & -21.0\% &  -7.9\% &   17.5\% &    15.8\% &    11.8\% &   23.9\% \\
        \textbf{PI} & -10.6\% & -47.1\% & -24.5\% &  -7.0\% &   25.2\% &    13.4\% &    24.3\% &   15.1\% \\
        \textbf{PR} &  -2.2\% & -37.2\% & -22.8\% & -11.5\% &    1.5\% &    -0.7\% &     1.4\% &   -2.0\% \\
        \textbf{RJ} &  -6.7\% & -43.6\% & -25.6\% & -17.2\% &   -8.6\% &   -10.4\% &   -17.7\% &  -17.7\% \\
        \textbf{RN} & -14.9\% & -47.3\% & -39.7\% & -23.3\% &   10.6\% &     8.1\% &    24.7\% &   31.9\% \\
        \textbf{RO} &  -3.0\% & -25.2\% & -10.0\% &  -4.7\% &   17.1\% &     2.1\% &    -0.1\% &   -4.8\% \\
        \textbf{RR} &  82.5\% &  90.6\% & 189.3\% & 370.4\% & 1480.1\% & - & - & - \\
        \textbf{RS} & -18.4\% & -42.0\% & -33.7\% & -20.1\% &   -7.9\% &   -17.9\% &   -15.0\% &  -23.9\% \\
        \textbf{SC} &  -8.2\% & -32.5\% & -18.8\% &  -6.3\% &   89.5\% &    12.3\% &    20.3\% &   19.7\% \\
        \textbf{SE} &  -4.2\% & -38.7\% &   6.2\% &  23.9\% &   16.0\% &    13.6\% &     7.4\% &   26.6\% \\
        \textbf{SP} &  -0.9\% & -32.9\% & -24.1\% &  -9.9\% &    3.0\% &     1.9\% &     4.6\% &    4.1\% \\
        \textbf{TO} & -13.4\% & -32.1\% & -18.9\% &   2.4\% &   18.4\% &    13.9\% &    23.7\% &   19.7\% \\
        \bottomrule
    \end{tabular}
\nota{Os dados de Roraima foram insuficientes para uma previsão correta e registraram valores com bastante ruído, fazendo com que a previsão de valores utilizando a regressão linear indicasse valores bastante pessimistas para os meses da pandemia. Foram mantidos os valores de março a julho e removidos os valores posteriores devido ao ruído. Da mesma forma, consideramos que o estado de Roraima não foi impactado negativamente pela pandemia}
\fdadospesquisa
\end{table}

\begin{table}[htb]
\centering
\caption{Diferença entre valores totais transacionados previstos e obtidos por UF no período da pandemia}
\label{tab:pandemia:variacao-total-por-uf}
    \begin{tabular}{l|r}
        \toprule
        \textbf{UF} & Variação total \\
        \midrule
        \textbf{AC} &   6.2\% \\
        \textbf{AL} &  -5.5\% \\
        \textbf{AM} & -15.7\% \\
        \textbf{AP} &   1.3\% \\
        \textbf{BA} &  -1.8\% \\
        \textbf{CE} &  -5.7\% \\
        \textbf{DF} &   5.1\% \\
        \textbf{ES} &  -4.9\% \\
        \textbf{GO} &  -0.4\% \\
        \textbf{MA} &  -0.7\% \\
        \textbf{MG} &  -3.1\% \\
        \textbf{MS} &  -2.9\% \\
        \textbf{MT} &   0.4\% \\
        \textbf{PA} &  -4.6\% \\
        \textbf{PB} &   5.9\% \\
        \textbf{PE} &  -1.7\% \\
        \textbf{PI} &  -0.7\% \\
        \textbf{PR} &  -8.9\% \\
        \textbf{RJ} & -18.4\% \\
        \textbf{RN} &  -6.1\% \\
        \textbf{RO} &  -3.4\% \\
        \textbf{RR} & 484.7\% \\
        \textbf{RS} & -22.1\% \\
        \textbf{SC} &  10.0\% \\
        \textbf{SE} &   6.6\% \\
        \textbf{SP} &  -6.6\% \\
        \textbf{TO} &   2.5\% \\
        \bottomrule
    \end{tabular}
\fdadospesquisa
\end{table}

\begin{table}[htb]
\centering
\caption{Quantidade de UFs impactadas negativamente pela pandemia em relação às variações mensais ou totais}
\label{tab:pandemia:impacto-por-uf}
    \begin{tabular}{l|r|r}
        \toprule
        Impacto & Mensal & Total \\
        \midrule
        Sim & 24 & 18 \\
        Não &  3 &  9 \\
        \bottomrule
    \end{tabular}
\fdadospesquisa
\end{table}

\section{Cenário setorial}

\begin{figure}[htb]
	\centering
    \caption{Valor total transacionado por seção no período analisado}
    \includegraphics[scale=0.7]{images/base-de-dados-15.1-valor-total-por-secao.png}
    \label{fig:pandemia:descritiva-15.1-valor-total-por-secao}
    \fdadospesquisa
\end{figure}

\begin{table}[htb]
\centering
\caption{Diferença entre valores mensais transacionados previstos e obtidos por seção no período da pandemia}
\label{tab:pandemia:variacao-mensal-por-secao}
    \begin{tabular}{c|r|r|r|r|r|r|r|r}
        \toprule
        \textbf{Seção} & Março & Abril & Maio & Junho & Julho & Agosto & Setembro & Outubro \\
        \midrule
        \textbf{A} & -15.4\% & -24.4\% & -14.2\% & -10.8\% &   6.9\% &   8.2\% &  32.9\% &  28.6\% \\
        \textbf{B} &   2.8\% &  12.8\% &  19.1\% &  32.7\% &  50.6\% &  51.0\% &  58.5\% &  64.6\% \\
        \textbf{C} &  -4.2\% & -39.0\% & -23.6\% &  -5.8\% &  16.0\% &  11.7\% &  15.5\% &  16.2\% \\
        \textbf{D} &  -1.4\% &  -6.1\% & -16.7\% & -14.4\% & -12.8\% & -39.3\% & -62.2\% & -64.5\% \\
        \textbf{E} &  10.8\% & -32.0\% & -17.3\% &  17.1\% &  65.5\% & 125.2\% &  -1.3\% &  15.3\% \\
        \textbf{F} &  -4.7\% & -30.0\% & -26.8\% & -20.0\% &  -8.1\% & -13.8\% & -10.0\% & -15.0\% \\
        \textbf{G} &  -5.9\% & -36.8\% & -23.0\% & -11.6\% &   4.0\% &  -2.3\% &  -0.2\% &  -2.3\% \\
        \textbf{H} & -16.2\% & -40.0\% & -39.2\% & -24.5\% & -17.0\% & -15.5\% &  -7.0\% &  -9.7\% \\
        \textbf{I} & -25.8\% & -53.7\% & -51.7\% & -42.2\% & -33.1\% & -33.4\% & -27.9\% & -25.3\% \\
        \textbf{J} &  24.3\% & -61.6\% & -47.4\% & -32.3\% &  -6.4\% &  -5.3\% & -19.3\% &   5.4\% \\
        \textbf{K} &   4.6\% &  15.1\% &  12.6\% &  13.2\% &  14.2\% & -72.6\% & -64.8\% & -66.1\% \\
        \textbf{L} & -28.3\% & -34.7\% & -56.8\% & -51.3\% & -49.2\% & -29.1\% & -34.2\% & -57.0\% \\
        \textbf{M} & -12.2\% & -12.2\% &  57.9\% &  71.4\% & -33.2\% & -32.2\% & -16.3\% & -12.8\% \\
        \textbf{N} &  -1.6\% & -52.1\% & -42.2\% & -10.0\% &  28.9\% &  -4.7\% &   5.7\% &   1.9\% \\
        \textbf{O} &  53.8\% &  20.7\% &   2.8\% &  77.7\% &  11.1\% &  -4.6\% &  27.2\% & -28.1\% \\
        \textbf{P} & -12.0\% & -56.8\% & -44.6\% & -33.5\% & -22.1\% & -34.0\% & -31.3\% & -39.5\% \\
        \textbf{Q} &  33.1\% & -10.5\% &  -6.4\% &  -7.2\% &   6.7\% &  -3.9\% &   2.7\% & -11.2\% \\
        \textbf{R} & -32.2\% & -68.9\% & -69.6\% & -50.6\% & -42.7\% & -41.5\% & -18.5\% & -18.3\% \\
        \textbf{S} &  -5.4\% & -32.9\% & -24.7\% & -22.5\% & -10.4\% & -19.5\% &   2.0\% &  -0.2\% \\
        \textbf{T} & -40.5\% & -38.2\% & -61.9\% & -48.5\% & -47.5\% & -54.7\% & -54.2\% & -27.6\% \\
        \bottomrule
    \end{tabular}
\fdadospesquisa
\end{table}

\begin{table}[htb]
\centering
\caption{Diferença entre valores totais transacionados previstos e obtidos por seção no período da pandemia}
\label{tab:pandemia:variacao-total-por-secao}
    \begin{tabular}{l|r}
        \toprule
        \textbf{Seção} & Variação total \\
        \midrule
        \textbf{A} &   2.4\% \\
        \textbf{B} &  36.0\% \\
        \textbf{C} &  -1.3\% \\
        \textbf{D} & -27.0\% \\
        \textbf{E} &  23.5\% \\
        \textbf{F} & -16.0\% \\
        \textbf{G} &  -9.5\% \\
        \textbf{H} & -20.8\% \\
        \textbf{I} & -36.5\% \\
        \textbf{J} & -17.4\% \\
        \textbf{K} & -17.7\% \\
        \textbf{L} & -42.8\% \\
        \textbf{M} &   0.5\% \\
        \textbf{N} &  -9.1\% \\
        \textbf{O} &  20.8\% \\
        \textbf{P} & -34.2\% \\
        \textbf{Q} &   0.3\% \\
        \textbf{R} & -42.7\% \\
        \textbf{S} & -14.0\% \\
        \textbf{T} & -46.6\% \\
        \bottomrule
    \end{tabular}
\fdadospesquisa
\end{table}

\begin{table}[htb]
\centering
\caption{Quantidade de seções impactadas negativamente pela pandemia em relação às variações mensais ou totais}
\label{tab:pandemia:impacto-por-secao}
    \begin{tabular}{l|r|r}
        \toprule
        Impacto & Mensal & Total \\
        \midrule
        Sim & 17 & 14 \\
        Não &  3 &  6 \\
        \bottomrule
    \end{tabular}
\fdadospesquisa
\end{table}

\begin{table}[htb]
\centering
\caption{Quantidade de CNAEs impactadas negativamente pela pandemia em relação às variações mensais ou totais}
\label{tab:pandemia:impacto-por-cnae}
    \begin{tabular}{l|r|r}
        \toprule
        Impacto & Mensal & Total \\
        \midrule
        Sim & 852 & 600 \\
        Não & 273 & 525 \\
        \bottomrule
    \end{tabular}
\fdadospesquisa
\end{table}


\chapter{Resultados}
\label{chapter:resultados}

Neste capítulo, serão apresentados os dados 

\section{Caracterização de redes}
\label{section:metricas-redes}

\subsection{Redes UF-UF}

Neste capítulo, faremos uma caracterização dos dados a partir de diferentes modelagems utilizando redes complexas.

\subsection{Redes UF-UF}
\label{section:metricas-redes:uf}

Nesta seção, será apresentada a caracterização de uma rede entre unidades federativas brasileiras. Essa rede foi montada utilizando os dados de transações entre janeiro de 2019 e dezembro de 2019. Cada vértice representa uma UF e cada aresta representa o valor total transacionado no período entre duas UFs. Nesta modelagem consideramos um grafo não-direcionado cíclico. Cada vértice possui um atributo chamado \textbf{região} indicando a região a que pertence aquela UF. Cada aresta além do atributo contendo o valor total transacional, e um atributo chamado \textbf{distância} correspondente ao inverso do valor total transacionado. A Tabela~\ref{tab:metricas-redes:grafo-por-uf} apresenta as métricas do grafo para modelagem.

\begin{table}[htb]
\centering
\caption{Métricas do grafo de transações entre UFs}
\label{tab:metricas-redes:grafo-por-uf}
    \begin{tabular}{l|r}
    \toprule
    Métrica &  Valor \\
    \midrule
    Quantidade de nós         &  27      \\
    Quantidade de arestas     & 377      \\
    Diâmetro                  &   2      \\
    Raio                      &   1      \\
    Grau médio                &  26.9259 \\
    Densidade                 &   1.0740 \\
    Transitividade            &   0.9971 \\
    Assortatividade de grau   &  -0.0356 \\
    Assortatividade de região &  -0.0025 \\
    \bottomrule
    \end{tabular}
\fdadospesquisa
\end{table}

É possível notar que o grafo entre UFs é quase completo, existindo arestas entre todas as UFs com uma exceção: o estado do Acre não possui conexão com o estado do Amapá. Por conta disso, temos um grafo de alta densidade e transitividade, e baixa assortatividade de grau e de região. A Tabela~\ref{tab:metricas-redes:grafo-por-cnae-especificas1} nos dá uma ideia melhor da topologia do grafo. Para essas métricas foi considerada a formulação usando arestas ponderadas. Para os caminhos mínimos, foi utilizada a distância entre UFs.

\begin{table}[htb]
\centering
\caption{Métricas do grafo de transações entre UFs segmentadas por UF}
\label{tab:metricas-redes:grafo-por-uf-especificas1}
    \begin{tabular}{l|rrrrrr}
    \toprule
    UF & Grau & \shortstack{Coeficiente\\de agrupamento} &  \shortstack{Caminho mínimo\\ médio poderado} & \shortstack{Centralidade\\de informação} &  \shortstack{Centralidade\\de autovalor} &  \textit{PageRank} \\
    \midrule
    AC & 26 & 0.0003 & $2.61e^{-09}$ & $4.51e^{+07}$ & 0.0016 & 0.0077 \\
    AL & 27 & 0.0008 & $1.07e^{-09}$ & $1.15e^{+08}$ & 0.0057 & 0.0112 \\
    AM & 27 & 0.0045 & $5.06e^{-10}$ & $2.33e^{+08}$ & 0.0714 & 0.0468 \\
    AP & 26 & 0.0005 & $2.66e^{-09}$ & $5.57e^{+07}$ & 0.0016 & 0.0077 \\
    BA & 27 & 0.0030 & $5.27e^{-10}$ & $2.16e^{+08}$ & 0.0538 & 0.0348 \\
    CE & 27 & 0.0019 & $6.39e^{-10}$ & $1.81e^{+08}$ & 0.0199 & 0.0185 \\
    DF & 27 & 0.0015 & $5.74e^{-10}$ & $1.83e^{+08}$ & 0.0290 & 0.0162 \\
    ES & 27 & 0.0033 & $5.03e^{-10}$ & $2.24e^{+08}$ & 0.0801 & 0.0334 \\
    GO & 27 & 0.0037 & $5.17e^{-10}$ & $2.25e^{+08}$ & 0.0634 & 0.0401 \\
    MA & 27 & 0.0012 & $8.34e^{-10}$ & $1.46e^{+08}$ & 0.0093 & 0.0131 \\
    MG & 27 & 0.0054 & $4.82e^{-10}$ & $2.39e^{+08}$ & 0.1696 & 0.0744 \\
    MS & 27 & 0.0015 & $5.96e^{-10}$ & $1.75e^{+08}$ & 0.0249 & 0.0153 \\
    MT & 27 & 0.0019 & $6.21e^{-10}$ & $1.85e^{+08}$ & 0.0219 & 0.0204 \\
    PA & 27 & 0.0021 & $5.89e^{-10}$ & $1.90e^{+08}$ & 0.0261 & 0.0216 \\
    PB & 27 & 0.0014 & $7.46e^{-10}$ & $1.58e^{+08}$ & 0.0124 & 0.0140 \\
    PE & 27 & 0.0029 & $5.59e^{-10}$ & $2.14e^{+08}$ & 0.0339 & 0.0325 \\
    PI & 27 & 0.0007 & $1.37e^{-09}$ & $9.40e^{+07}$ & 0.0039 & 0.0084 \\
    PR & 27 & 0.0049 & $4.85e^{-10}$ & $2.37e^{+08}$ & 0.1416 & 0.0630 \\
    RJ & 27 & 0.0051 & $4.82e^{-10}$ & $2.38e^{+08}$ & 0.1596 & 0.0659 \\
    RN & 27 & 0.0011 & $8.10e^{-10}$ & $1.33e^{+08}$ & 0.0097 & 0.0125 \\
    RO & 27 & 0.0010 & $8.09e^{-10}$ & $1.45e^{+08}$ & 0.0100 & 0.0138 \\
    RR & 27 & 0.0003 & $1.83e^{-09}$ & $4.51e^{+07}$ & 0.0009 & 0.0067 \\
    RS & 27 & 0.0038 & $5.16e^{-10}$ & $2.26e^{+08}$ & 0.0680 & 0.0416 \\
    SC & 27 & 0.0042 & $4.97e^{-10}$ & $2.31e^{+08}$ & 0.0971 & 0.0474 \\
    SE & 27 & 0.0006 & $1.64e^{-09}$ & $9.23e^{+07}$ & 0.0025 & 0.0091 \\
    SP & 27 & 0.0119 & $4.59e^{-10}$ & $2.52e^{+08}$ & 0.9426 & 0.3101 \\
    TO & 27 & 0.0011 & $8.96e^{-10}$ & $1.42e^{+08}$ & 0.0085 & 0.0127 \\
    \bottomrule
    \end{tabular}
\fdadospesquisa
\end{table}

O estado de Sâo Paulo se sobressai com maiores valores de coeficiente de agrupamento, centralidade de autovalor, e pagerank. Isso provavelmente se deve ao maior valor transacionado entre este estado e todos os outros, uma vez que o estado é conhecido por ser peça importante no cenário econômico brasileiro. Observando outros estados da região sudeste, como Rio de Janeiro de Minas Gerais, estes também tem valores importantes para essas métricas. Sendo assim, apesar de uma topologia simples ela se mostra razoavelmente eficaz em mostrar a importância de cada estado em relação às relações econômicas entre UFs brasileiras.

Nesta análise, não consideramos as métricas de centralidade de intermediação, caminho mínimo médio, e excentricidade, uma vez que o grafo é quase completo e essas medidas refletiriam apenas o grau de cada vértice.

\subsection{Redes Seção-Seção}
\label{section:metricas-redes:secao}

Nesta seção, será apresentada a caracterização de uma rede entre seções de CNAE. Essa rede foi montada utilizando os dados de transações entre janeiro de 2019 e dezembro de 2019, como anteriormente. Cada vértice representa uma seção e cada aresta representa o valor total transacionado no período entre duas seções. Nesta modelagem consideramos um grafo não-direcionado cíclico. Cada aresta além do atributo contendo o valor total transacional, e um atributo chamado \textbf{distância} correspondente ao inverso do valor total transacionado. A Tabela~\ref{tab:metricas-redes:grafo-por-secao} apresenta as métricas do grafo para modelagem.

\begin{table}[htb]
\centering
\caption{Métricas do grafo de transações entre seções}
\label{tab:metricas-redes:grafo-por-secao}
    \begin{tabular}{l|r}
    \toprule
    Métrica &  Valor \\
    \midrule
    Quantidade de nós       &   20      \\
    Quantidade de arestas   &  186      \\
    Diâmetro                &    2      \\
    Raio                    &    1      \\
    Grau médio              &   17.6000 \\
    Densidade               &    0.9789 \\
    Transitividade          &    0.9618 \\
    Assortatividade de grau &   -0.1433 \\
    \bottomrule
    \end{tabular}
\fdadospesquisa
\end{table}

Podemos observar que este grafo não é tão completo quanto o anterior mas também é bastante completo, tendo altos valores de grau médio, densidade, e transitividade, assim como baixos valores de raio e diâmetro. A assortatividade de grau indica leve tendência à assortatividade negativa, porém dada a topologia da rede já é algo também esperado uma vez que a rede é bastante conectada mas existe alguma variância entre os graus de cada vértice como veremos.

As Tabelas~\ref{tab:metricas-redes:grafo-por-secao-especificas1} e \ref{tab:metricas-redes:grafo-por-secao-especificas2} detalham as métricas agora por seção, sendo novamente considerada a formulação usando arestas ponderadas. Para os caminhos mínimos ponderados, foi utilizada a distância entre seções.

\begin{table}[htb]
\centering
\caption{Métricas do grafo de transações entre seções segmentadas por seção (Parte 1)}
\label{tab:metricas-redes:grafo-por-secao-especificas1}
    \begin{tabular}{l|rrrr}
    \toprule
    Seção &  Grau &  \shortstack{Coeficiente\\de agrupamento} &  \shortstack{Caminho\\mínimo médio} &  \shortstack{Caminho mínimo\\médio ponderado} \\
    \midrule
    A &  19 &          0.0006 &     1 &  $1.635e^{-06}$ \\
    B &  19 &          0.0007 &     1 &  $1.635e^{-06}$ \\
    C &  20 &          0.0047 &  0.95 &  $1.634e^{-06}$ \\
    D &  19 &          0.0010 &     1 &  $1.634e^{-06}$ \\
    E &  19 &          0.0003 &     1 &  $1.635e^{-06}$ \\
    F &  19 &          0.0005 &     1 &  $1.635e^{-06}$ \\
    G &  20 &          0.0048 &  0.95 &  $1.634e^{-06}$ \\
    H &  19 &          0.0010 &     1 &  $1.634e^{-06}$ \\
    I &  19 &          0.0008 &     1 &  $1.634e^{-06}$ \\
    J &  19 &          0.0009 &     1 &  $1.634e^{-06}$ \\
    K &  17 &          0.0005 &   1.1 &  $1.635e^{-06}$ \\
    L &  15 &  $7.119e^{-05}$ &   1.2 &  $1.648e^{-06}$ \\
    M &  19 &          0.0002 &     1 &  $1.635e^{-06}$ \\
    N &  19 &          0.0005 &     1 &  $1.635e^{-06}$ \\
    O &  18 &          0.0005 &  1.05 &  $1.634e^{-06}$ \\
    P &  18 &          0.0001 &  1.05 &  $1.637e^{-06}$ \\
    Q &  19 &          0.0008 &     1 &  $1.634e^{-06}$ \\
    R &  15 &  $4.308e^{-05}$ &   1.1 &  $1.646e^{-06}$ \\
    S &  19 &          0.0002 &     1 &  $1.635e^{-06}$ \\
    T &   1 &  $1.285e^{-05}$ &   1.8 &  $3.102e^{-05}$ \\
    \bottomrule
    \end{tabular}
\fdadospesquisa
\end{table}

Podemos observar que assim como anteriormente, as seções C e G se sobressaem com altos valores de agrupamento e centralidade, indicando influencia dos altos valores transacionados por essas seções com outras regiões. A seção T se sobressai com uma baixa conectividade com outras seções, e as seções L e R são bastante influenciadas pelos baixos valores transacionados.

\begin{table}[htb]
\centering
\caption{Métricas do grafo de transações entre seções segmentadas por seção (Parte 2)}
\label{tab:metricas-redes:grafo-por-secao-especificas2}
    \begin{tabular}{l|rrrrr}
    \toprule
    Seção & \shortstack{Centralidade\\de informação} &  \shortstack{Centralidade\\de autovalor} &   \shortstack{Centralidade\\de intermediação} & \shortstack{Centralidade\\de proximidade} & PageRank \\
    \midrule
    A & $3.907e^{+04}$ &  0.0116 &  0.0020 &  0.95 &  0.0131 \\
    B & $3.907e^{+04}$ &  0.0090 &  0.0020 &  0.95 &  0.0132 \\
    C & $3.907e^{+04}$ &  0.6961 &  0.0517 &  1    &  0.3633 \\
    D & $3.907e^{+04}$ &  0.0139 &  0.0020 &  0.95 &  0.0457 \\
    E & $3.906e^{+04}$ &  0.0038 &  0.0020 &  0.95 &  0.0093 \\
    F & $3.906e^{+04}$ &  0.0067 &  0.0020 &  0.95 &  0.0110 \\
    G & $3.907e^{+04}$ &  0.7167 &  0.0517 &  1    &  0.3954 \\
    H & $3.907e^{+04}$ &  0.0213 &  0.0020 &  0.95 &  0.0171 \\
    I & $3.907e^{+04}$ &  0.0123 &  0.0020 &  0.95 &  0.0143 \\
    J & $3.907e^{+04}$ &  0.0173 &  0.0020 &  0.95 &  0.0169 \\
    K & $3.907e^{+04}$ &  0.0032 &  0      &  0.86 &  0.0147 \\
    L & $3.887e^{+04}$ &  0.0001 &  0      &  0.79 &  0.0075 \\
    M & $3.905e^{+04}$ &  0.0020 &  0.0020 &  0.95 &  0.0090 \\
    N & $3.906e^{+04}$ &  0.0066 &  0.0020 &  0.95 &  0.0105 \\
    O & $3.907e^{+04}$ &  0.0085 &  0.0003 &  0.90 &  0.0113 \\
    P & $3.903e^{+04}$ &  0.0006 &  0.0003 &  0.90 &  0.0078 \\
    Q & $3.907e^{+04}$ &  0.0177 &  0.0020 &  0.95 &  0.0156 \\
    R & $3.881e^{+04}$ &  0.0001 &  0      &  0.86 &  0.0075 \\
    S & $3.905e^{+04}$ &  0.0016 &  0.0020 &  0.95 &  0.0084 \\
    T & $2.058e^{+03}$ &  4.826e &  0      &  0.53 &  0.0075 \\
    \bottomrule
    \end{tabular}
\fdadospesquisa
\end{table}

\subsection{Redes CNAE-CNAE}
\label{section:metricas-redes:cnae}

Nesta seção, vamos repetir a análise agora para a caracterização de uma rede entre CNAEs. Essa rede foi montada utilizando os dados de transações entre janeiro de 2019 e dezembro de 2019. Cada vértice representa um CNAE e cada aresta representa o valor total transacionado no período entre dois CNAEs. Nesta modelagem consideramos um grafo não-direcionado cíclico. Cada vértice possui um atributo chamado \textbf{seção} indicando a seção a que pertence aquele CNAE. Cada aresta além do atributo contendo o valor total transacional, e um atributo chamado \textbf{distância} correspondente ao inverso do valor total transacionado. A Tabela~\ref{tab:metricas-redes:grafo-por-cnae} apresenta as métricas do grafo para modelagem.

\begin{table}[htb]
\centering
\caption{Métricas do grafo de transações entre CNAEs}
\label{tab:metricas-redes:grafo-por-cnae}
    \begin{tabular}{l|r}
    \toprule
    Métrica &  Valor \\
    \midrule
    Quantidade de nós         &   1125      \\
    Quantidade de arestas     & 135607      \\
    Diâmetro                  &      4      \\
    Raio                      &      2      \\
    Grau médio                &    241.0791 \\
    Densidade                 &      0.2145 \\
    Transitividade            &      0.5293 \\
    Assortatividade de grau   &     -0.3449 \\
    Assortatividade de seção &      0.0015 \\
    \bottomrule
    \end{tabular}
\fdadospesquisa
\end{table}



\begin{table}[htb]
\centering
\caption{Métricas do grafo de transações entre CNAEs segmentadas por CNAE (Parte 1)}
\label{tab:metricas-redes:grafo-por-cnae-especificas1}
\begin{tabular}{l|rrrrr}
\toprule
Métrica & Grau & \shortstack{Coeficiente\\de agrupamento} & Excentricidade & \shortstack{Caminho\\mínimo médio} & \shortstack{Caminho mínimo\\médio ponderado} \\
\midrule
Média         &  241.1 &   $1.471e^{-05}$ & 2.917 & 1.8160 & $1.140e^{-05}$ \\
Desvio Padrão &  213.7 &   $1.946e^{-05}$ & 0.309 & 0.2289 & $6.641e^{-05}$ \\
Mínimo        &    1   &                0 &     2 & 1.1030 & $5.708e^{-06}$ \\
25º percentil &   66   &   $4.441e^{-06}$ &     3 & 1.6800 & $5.722e^{-06}$ \\
50º percentil &  183   &   $9.605e^{-06}$ &     3 & 1.8470 & $5.776e^{-06}$ \\
75º percentil &  365   &   $1.790e^{-05}$ &     3 & 1.9640 & $6.351e^{-06}$ \\
Máximo        & 1009   &           0.0003 &     4 & 2.7970 &         0.0019 \\
\bottomrule
\end{tabular}
\fdadospesquisa
\end{table}

\begin{table}[htb]
\centering
\caption{Métricas do grafo de transações entre CNAEs segmentadas por CNAE (Parte 2)}
\label{tab:metricas-redes:grafo-por-cnae-especificas2}
\begin{tabular}{l|rrrrr}
\toprule
Métrica & \shortstack{Centralidade\\de informação} & \shortstack{Centralidade\\de autovalor} & \shortstack{Centralidade de\\intermediação} & \shortstack{Centralidade\\de proximidade} & PageRank \\
\midrule
Média         & 225.9    &         0.0020 &         0.0007 & 0.5598 & 0.0008 \\
Desvio Padrão &  47.49   &         0.0298 &         0.0019 & 0.0769 & 0.0021 \\
Mínimo        &   0.4702 & $3.198e^{-12}$ &              0 & 0.3572 & 0.0001 \\
25º percentil & 235.5    & $3.452e^{-07}$ & $3.047e^{-06}$ & 0.5086 & 0.0001 \\
50º percentil & 245.4    & $5.856e^{-06}$ & $5.415e^{-05}$ & 0.5409 & 0.0002 \\
75º percentil & 246.3    & $7.721e^{-05}$ &         0.0005 & 0.5947 & 0.0007 \\
Máximo        & 46.4     &         0.8161 &         0.0287 & 0.9057 & 0.0316 \\
\bottomrule
\end{tabular}
\fdadospesquisa
\end{table}

\section{Alterações topológicas}
\label{section:alteracoes-topologicas}

\subsection{Redes UF-UF}
\label{section:alteracoes-topologicas:uf}

\begin{table}[htb]
\centering
\caption{Variação percentual das métricas de redes mês a mês para o grafo de UFs (Parte 1)}
\label{tab:metricas-redes-pandemia:grafo-mensal-por-uf1}
\begin{tabular}{l|rrrr}
\toprule
Mês & Densidade & Transitividade & Grau médio & \shortstack{Grau médio\\ponderado} \\
\midrule
01/2019 & - & - & - & - \\
02/2019 &  0.54\% &  0.55\% &  0.54\% &  9.88\% \\
03/2019 &  0.00\% &  0.17\% &  0.00\% & 25.54\% \\
04/2019 &  0.54\% &  0.57\% &  0.54\% & 23.86\% \\
05/2019 &  0.81\% &  0.81\% &  0.81\% & 36.91\% \\
06/2019 &  0.27\% &  0.32\% &  0.27\% & 23.88\% \\
07/2019 &  0.54\% &  0.57\% &  0.54\% & 28.36\% \\
08/2019 &  0.54\% &  0.57\% &  0.54\% & 35.37\% \\
09/2019 &  0.27\% &  0.36\% &  0.27\% & 32.62\% \\
10/2019 &  0.81\% &  0.82\% &  0.81\% & 42.89\% \\
11/2019 &  0.27\% &  0.34\% &  0.27\% & 43.93\% \\
12/2019 &  0.27\% &  0.36\% &  0.27\% & 29.66\% \\
01/2020 &  0.81\% &  0.83\% &  0.81\% & 28.84\% \\
02/2020 &  0.27\% &  0.44\% &  0.27\% & 32.22\% \\
03/2020 &  0.54\% &  0.59\% &  0.54\% & 36.09\% \\
04/2020 & -0.27\% & -0.07\% & -0.27\% & -7.91\% \\
05/2020 & -0.27\% & -0.18\% & -0.27\% & 12.47\% \\
06/2020 &  0.00\% &  0.11\% &  0.00\% & 34.61\% \\
07/2020 &  0.54\% &  0.57\% &  0.54\% & 62.51\% \\
08/2020 &  0.54\% &  0.59\% &  0.54\% & 55.16\% \\
09/2020 &  1.08\% &  1.09\% &  1.08\% & 60.32\% \\
10/2020 &  0.00\% &  0.17\% &  0.00\% & 61.19\% \\
\bottomrule
\end{tabular}
\fdadospesquisa
\end{table}

\begin{table}[htb]
\centering
\caption{Variação das métricas de redes mês a mês para o grafo de UFs (Parte 2)}
\label{tab:metricas-redes-pandemia:grafo-mensal-por-uf2}
\begin{tabular}{l|rrrr}
\toprule
Mês & \shortstack{Assortatividade\\de grau} & \shortstack{Assortatividade\\de região} & \shortstack{Caminho mínimo\\médio} & \shortstack{Caminho mínimo\\médio ponderado} \\
\midrule
01/2019 & - & - & - & - \\
02/2019 & -15.60\% & -50.13\% & -0.55\% &  -8.47\% \\
03/2019 &   0.05\% &  -0.00\% &  0.00\% & -25.12\% \\
04/2019 &   7.91\% & -15.39\% & -0.55\% & -22.23\% \\
05/2019 & -31.31\% & -85.53\% & -0.83\% & -25.07\% \\
06/2019 &  21.99\% &  13.35\% & -0.27\% &   2.44\% \\
07/2019 & -17.07\% & -57.33\% & -0.55\% &  -1.69\% \\
08/2019 &   7.91\% & -57.33\% & -0.55\% & -11.02\% \\
09/2019 &   4.73\% & -28.82\% & -0.27\% &  -2.55\% \\
10/2019 &  -6.87\% & -43.83\% & -0.83\% & -11.15\% \\
11/2019 &   8.56\% &  13.35\% & -0.27\% & -14.48\% \\
12/2019 &   4.73\% & -70.86\% & -0.27\% &   2.50\% \\
01/2020 & -25.43\% & -43.83\% & -0.83\% &   1.69\% \\
02/2020 & -18.97\% & -28.82\% & -0.27\% &  -5.58\% \\
03/2020 &  -7.22\% & -15.39\% & -0.55\% &  -4.69\% \\
04/2020 &   1.98\% & -13.38\% &  0.27\% &  36.70\% \\
05/2020 &  37.79\% &  29.14\% &  0.27\% &   2.72\% \\
06/2020 &  19.66\% & -42.28\% &  0.00\% & -15.66\% \\
07/2020 &   7.91\% & -15.39\% & -0.55\% & -32.97\% \\
08/2020 &  -7.22\% & -15.39\% & -0.55\% & -35.56\% \\
09/2020 & -16.88\% & -71.96\% & -1.11\% & -37.64\% \\
10/2020 &   0.05\% & -42.28\% &  0.00\% & -30.06\% \\
\bottomrule
\end{tabular}
\fdadospesquisa
\end{table}

\begin{table}[htb]
\centering
\caption{Variação das métricas de rede a cada trimestre para o grafo de UFs (Parte 1)}
\label{tab:metricas-redes-pandemia:grafo-trimestral-por-uf1}
\begin{tabular}{l|rrrr}
\toprule
Mês & Densidade & Transitividade & Grau médio & \shortstack{Grau médio\\ponderado} \\
\midrule
2020/1 & - & - & - & - \\
2020/2 & -0.80\% & -0.78\% & -0.80\% & -14.60\% \\
2020/3 &  0.27\% &  0.22\% &  0.27\% &  20.35\% \\
\bottomrule
\end{tabular}
\fdadospesquisa
\end{table}

\begin{table}[htb]
\centering
\caption{Variação das métricas de rede a cada trimestre para o grafo de UFs (Parte 2)}
\label{tab:metricas-redes-pandemia:grafo-trimestral-por-uf2}
\begin{tabular}{l|rrrr}
\toprule
Mês & \shortstack{Assortatividade\\de grau} & \shortstack{Assortatividade\\de região} & \shortstack{Caminho mínimo\\médio} & \shortstack{Caminho mínimo\\médio ponderado} \\
\midrule
2020/1 & - & - & - & - \\
2020/2 &  77.29\% &   1.84\% &  0.85\% &   6.63\% \\
2020/3 & -34.35\% & -83.36\% & -0.28\% & -33.74\% \\
\bottomrule
\end{tabular}
\fdadospesquisa
\end{table}

\begin{table}[htb]
\centering
\caption{Diferença entre as métricas de rede para UFs afetadas e não afetadas}
\label{tab:metricas-redes-pandemia:diferenca-afetadas-por-uf}
\begin{tabular}{l|rrrrrrr}
\toprule
Métrica & Média & \shortstack{Desvio\\padrão} & Mínimo & P25 & Mediana & P75 & Máximo \\
\midrule
Grau médio                     & -1.85\% &   2.51\% &  -0.26\% & -3.70\% &  0.00\% &  0.00\% &    0.00\% \\ \hline
\shortstack[l]{Coeficiente de\\agrupamento}     &  0.33\% &   4.71\% &   2.79\% & -2.34\% & -0.83\% &  6.67\% &  -14.51\% \\ \hline
Excentricidade                 & -4.17\% & -20.41\% &   0.00\% &  0.00\% &  0.00\% &  0.00\% & -100.00\% \\ \hline
\shortstack[l]{Caminho mínimo\\médio}           &  1.85\% &   2.46\% &   0.00\% &  0.00\% &  0.00\% &  3.70\% &   -0.28\% \\ \hline
\shortstack[l]{Caminho mínimo\\médio ponderado} & -6.58\% &   0.14\% &  -2.01\% & -7.36\% & -0.36\% & -1.03\% &  -37.83\% \\ \hline
\shortstack[l]{Centralidade de\\proximidade}    & -1.71\% &   2.28\% &   0.25\% & -3.45\% &  0.00\% &  0.00\% &    0.00\% \\ \hline
\shortstack[l]{Centralidade de\\informação}     & -1.07\% &  -1.71\% &   9.93\% & -1.09\% &  0.16\% &  0.10\% &   -8.60\% \\ \hline
\shortstack[l]{Centralidade de\\autovalor}      & -3.84\% &  -1.85\% &  23.00\% & -0.99\% & -1.58\% & -4.25\% &  -20.99\% \\ \hline
\shortstack[l]{Centralidade de\\intermediação}  &  - &     - & 206.06\% &  0.00\% &  0.00\% &  0.00\% &    - \\ \hline
PageRank                       & -0.63\% &   0.02\% &   4.37\% & -1.23\% & -1.61\% & -1.01\% &   -5.26\% \\
\bottomrule
\end{tabular}
\fdadospesquisa
\end{table}

\subsection{Redes Seção-Seção}
\label{section:alteracoes-topologicas:secao}

\begin{table}[htb]
\centering
\caption{Variação percentual das métricas de redes mês a mês para o grafo de seções (Parte 1)}
\label{tab:metricas-redes-pandemia:grafo-mensal-por-secao1}
\begin{tabular}{l|rrrr}
\toprule
Mês & Densidade & Transitividade & Grau médio & \shortstack{Grau médio\\ponderado} \\
\midrule
01/2019 &  - &  - &  - &  - \\
02/2019 & -0.57\% & -0.55\% & -0.57\% &  9.89\%\\
03/2019 &  0.57\% &  0.31\% &  0.57\% & 25.54\%\\
04/2019 & -1.15\% & -1.20\% & -1.15\% & 23.86\%\\
05/2019 &  1.15\% &  0.15\% &  1.15\% & 36.92\%\\
06/2019 & -1.72\% & -1.33\% & -1.72\% & 23.89\%\\
07/2019 & -0.57\% & -1.00\% & -0.57\% & 28.37\%\\
08/2019 & -0.57\% & -0.63\% & -0.57\% & 35.37\%\\
09/2019 &  0.00\% &  0.18\% &  0.00\% & 32.63\%\\
10/2019 & -0.57\% &  0.13\% & -0.57\% & 42.89\%\\
11/2019 & -1.72\% & -1.25\% & -1.72\% & 43.94\%\\
12/2019 &  1.15\% &  0.85\% &  1.15\% & 29.66\%\\
01/2020 &  2.30\% &  1.87\% &  2.30\% & 28.85\%\\
02/2020 &  1.15\% &  0.46\% &  1.15\% & 32.22\%\\
03/2020 &  1.72\% &  0.98\% &  1.72\% & 36.10\%\\
04/2020 & -0.57\% & -0.38\% & -0.57\% & -7.92\%\\
05/2020 & -0.57\% & -0.47\% & -0.57\% & 12.47\%\\
06/2020 & -1.72\% & -0.72\% & -1.72\% & 34.62\%\\
07/2020 & -1.72\% & -1.97\% & -1.72\% & 62.52\%\\
08/2020 &  0.00\% & -0.15\% &  0.00\% & 55.16\%\\
09/2020 &  1.72\% &  1.24\% &  1.72\% & 60.33\%\\
10/2020 &  0.00\% & -0.60\% &  0.00\% & 61.20\%\\
\bottomrule
\end{tabular}
\fdadospesquisa
\end{table}

\begin{table}[htb]
\centering
\caption{Variação das métricas de redes mês a mês para o grafo de seções (Parte 2)}
\label{tab:metricas-redes-pandemia:grafo-mensal-por-secao2}
\begin{tabular}{l|rrrr}
\toprule
Mês & \shortstack{Assortatividade\\de grau} & \shortstack{Assortatividade\\de seção} & \shortstack{Caminho mínimo\\médio} & \shortstack{Caminho mínimo\\médio ponderado} \\
\midrule
01/2019 &  - &  - &  - &  - \\
02/2019 &   2.76\% &  -4.38\% &  0.45\% &  29.22\% \\
03/2019 &  -9.42\% &   2.43\% & -0.45\% & -11.35\% \\
04/2019 &   4.86\% &  -8.49\% &  0.90\% &   2.19\% \\
05/2019 &   1.47\% & -55.07\% & -0.45\% &   3.75\% \\
06/2019 &  16.41\% &  51.81\% &  0.90\% &   5.20\% \\
07/2019 &   1.48\% &  -6.63\% &  0.45\% & -15.91\% \\
08/2019 &   6.16\% &  -4.75\% &  0.45\% & -13.54\% \\
09/2019 &  -5.13\% &  61.37\% & -0.45\% & -10.76\% \\
10/2019 &  -1.58\% &  59.27\% &  0.00\% & -15.12\% \\
11/2019 &  10.43\% &  50.66\% &  0.90\% & -38.71\% \\
12/2019 &  -3.53\% &   8.42\% & -0.90\% & -41.20\% \\
01/2020 & -14.48\% &  16.07\% & -1.80\% & -25.39\% \\
02/2020 &  -1.24\% & -53.98\% & -0.45\% & -36.64\% \\
03/2020 &  -7.40\% & -50.23\% & -0.90\% &  22.58\% \\
04/2020 &  -4.91\% &  -5.13\% &  0.45\% &  28.02\% \\
05/2020 &  -8.57\% &  -6.63\% &  0.45\% & 114.33\% \\
06/2020 &  -3.99\% & -11.95\% &  1.35\% &  72.81\% \\
07/2020 &  -4.68\% & -80.36\% &  1.80\% &  65.47\% \\
08/2020 & -11.69\% & -64.36\% &  0.45\% &  60.28\% \\
09/2020 & -10.17\% & -49.15\% & -0.90\% &  56.97\% \\
10/2020 &  -6.61\% & -65.48\% &  0.45\% & -33.58\% \\
\bottomrule
\end{tabular}
\fdadospesquisa
\end{table}

\begin{table}[htb]
\centering
\caption{Variação das métricas de rede a cada trimestre para o grafo de seções (Parte 1)}
\label{tab:metricas-redes-pandemia:grafo-trimestral-por-secao1}
\begin{tabular}{l|rrrr}
\toprule
Mês & Densidade & Transitividade & Grau médio & \shortstack{Grau médio\\ponderado} \\
\midrule
2020/1 & - & - & - & - \\
2020/2 & -1.66\% & -0.92\% & -1.66\% & -14.60\% \\
2020/3 & -1.10\% & -0.59\% & -1.10\% &  20.35\% \\
\bottomrule
\end{tabular}
\fdadospesquisa
\end{table}

\begin{table}[htb]
\centering
\caption{Variação das métricas de rede a cada trimestre para o grafo de seções (Parte 2)}
\label{tab:metricas-redes-pandemia:grafo-trimestral-por-secao2}
\begin{tabular}{l|rrrr}
\toprule
Mês & \shortstack{Assortatividade\\de grau} & \shortstack{Assortatividade\\de seção} & \shortstack{Caminho mínimo\\médio} & \shortstack{Caminho mínimo\\médio ponderado} \\
\midrule
2020/1 & - & - & - & - \\
2020/2 & 11.16\% &  72.34\% & 0.93\% & 122.98\% \\
2020/3 & -2.87\% & -12.77\% & 0.93\% & 100.19\% \\
\bottomrule
\end{tabular}
\fdadospesquisa
\end{table}

\begin{table}[htb]
\centering
\caption{Diferença entre as métricas de rede para seções afetadas e não afetadas}
\label{tab:metricas-redes-pandemia:diferenca-afetadas-por-secao}
\begin{tabular}{l|rrrrrrr}
\toprule
Métrica & Média & \shortstack{Desvio\\padrão} & Mínimo & P25 & Mediana & P75 & Máximo \\
\midrule
Grau médio                     &  0.02\% & -0.04\% & 0.11\% & 0.05\% &  0.00\% &  0.00\% & -0.05\% \\ \hline
\shortstack[l]{Coeficiente de\\agrupamento}     &  0.16\% & -0.01\% & 0.38\% & 0.15\% &  0.11\% &  0.14\% &  0.07\% \\ \hline
Excentricidade                 &  0.00\% &  0.00\% & 0.00\% & 0.00\% &  0.00\% &  0.00\% &  0.00\% \\ \hline
\shortstack[l]{Caminho mínimo\\médio}           & -0.01\% & -0.03\% & 0.05\% & 0.00\% &  0.00\% & -0.04\% & -0.05\% \\ \hline
\shortstack[l]{Caminho mínimo\\médio ponderado} &  0.00\% & -0.00\% & 0.00\% & 0.00\% & -0.00\% & -0.00\% & -0.00\% \\ \hline
\shortstack[l]{Centralidade de\\proximidade}    &  0.01\% & -0.03\% & 0.05\% & 0.04\% &  0.00\% &  0.00\% & -0.05\% \\ \hline
\shortstack[l]{Centralidade de\\informação}     &  0.00\% & -0.00\% & 0.00\% & 0.00\% &  0.00\% & -0.00\% & -0.00\% \\ \hline
\shortstack[l]{Centralidade de\\autovalor}      &  0.32\% & -0.00\% & 0.56\% & 0.39\% &  0.20\% &  0.25\% &  0.28\% \\ \hline
\shortstack[l]{Centralidade de\\intermediação}  &  0.43\% &  0.12\% & 0.76\% & 0.34\% &  0.00\% &  0.46\% &  0.00\% \\ \hline
PageRank                       &  0.07\% & -0.04\% & 0.17\% & 0.07\% &  0.08\% &  0.11\% & -0.21\% \\
\bottomrule
\end{tabular}
\fdadospesquisa
\end{table}

\subsection{Redes CNAE-CNAE}
\label{section:alteracoes-topologicas:cnae}

\begin{table}[htb]
\centering
\caption{Variação percentual das métricas de redes mês a mês para o grafo de CNAEs (Parte 1)}
\label{tab:metricas-redes-pandemia:grafo-mensal-por-cnae1}
\begin{tabular}{l|rrrr}
\toprule
Mês & Densidade & Transitividade & Grau médio & \shortstack{Grau médio\\ponderado} \\
\midrule
01/2019 & - & - & - & - \\
02/2019 &  4.35\% &  1.75\% &  4.45\% &  9.79\% \\
03/2019 &  5.96\% &  2.29\% &  6.15\% & 25.32\% \\
04/2019 &  9.60\% &  3.67\% &  9.70\% & 23.75\% \\
05/2019 & 11.46\% &  4.22\% & 11.66\% & 36.67\% \\
06/2019 &  7.55\% &  2.71\% &  7.55\% & 23.89\% \\
07/2019 & 12.16\% &  4.16\% & 12.06\% & 28.48\% \\
08/2019 & 12.70\% &  4.64\% & 12.80\% & 35.25\% \\
09/2019 & 11.38\% &  4.45\% & 11.48\% & 32.51\% \\
10/2019 & 13.87\% &  5.36\% & 14.27\% & 42.38\% \\
11/2019 & 12.57\% &  4.84\% & 12.67\% & 43.81\% \\
12/2019 &  6.15\% &  2.35\% &  6.62\% & 29.09\% \\
01/2020 &  9.87\% &  3.64\% & 10.27\% & 28.39\% \\
02/2020 &  8.58\% &  3.01\% &  8.87\% & 31.87\% \\
03/2020 &  9.29\% &  3.80\% &  9.68\% & 35.62\% \\
04/2020 & -2.40\% & -0.24\% & -2.40\% & -7.92\% \\
05/2020 &  1.98\% &  1.67\% &  2.07\% & 12.37\% \\
06/2020 &  7.94\% &  3.99\% &  8.13\% & 34.38\% \\
07/2020 & 11.86\% &  5.21\% & 12.46\% & 61.65\% \\
08/2020 & 10.99\% &  4.48\% & 10.89\% & 55.30\% \\
09/2020 &  9.68\% &  4.69\% & 10.26\% & 59.47\% \\
10/2020 &  7.63\% &  3.96\% &  8.02\% & 60.62\% \\
\bottomrule
\end{tabular}
\fdadospesquisa
\end{table}

\begin{table}[htb]
\centering
\caption{Variação das métricas de redes mês a mês para o grafo de CNAEs (Parte 2)}
\label{tab:metricas-redes-pandemia:grafo-mensal-por-cnae2}
\begin{tabular}{l|rrrr}
\toprule
Mês & \shortstack{Assortatividade\\de grau} & \shortstack{Assortatividade\\de seção} & \shortstack{Caminho mínimo\\médio} & \shortstack{Caminho mínimo\\médio ponderado} \\
\midrule
01/2019 & - & - & - & - \\
02/2019 &  0.38\% &  -6.76\% & -0.72\% &  -9.37\% \\
03/2019 &  1.52\% & -22.04\% & -0.93\% &  35.22\% \\
04/2019 &  1.80\% &  -8.60\% & -1.49\% &  33.50\% \\
05/2019 &  2.69\% & -23.49\% & -1.83\% & -13.22\% \\
06/2019 &  1.02\% & -24.50\% & -1.15\% & -15.84\% \\
07/2019 &  1.76\% &   4.38\% & -1.77\% & -17.73\% \\
08/2019 &  1.77\% & -16.28\% & -1.64\% &   3.81\% \\
09/2019 &  0.86\% &   3.95\% & -1.65\% & -29.89\% \\
10/2019 &  1.55\% &  -4.69\% & -1.83\% & -35.86\% \\
11/2019 &  1.02\% &  -8.01\% & -1.83\% & -36.74\% \\
12/2019 &  0.52\% & -10.46\% & -1.08\% & -30.56\% \\
01/2020 &  2.06\% &  -6.52\% & -1.50\% & -28.02\% \\
02/2020 &  1.27\% &   9.35\% & -1.30\% & -23.95\% \\
03/2020 &  1.61\% &   1.73\% & -1.40\% & -23.14\% \\
04/2020 & -0.02\% &   8.10\% &  0.58\% &  -8.54\% \\
05/2020 &  1.14\% &  16.10\% & -0.27\% &   9.73\% \\
06/2020 &  2.35\% &   0.88\% & -1.13\% &  16.51\% \\
07/2020 &  3.60\% &  15.25\% & -1.66\% &   1.86\% \\
08/2020 &  1.83\% &  12.58\% & -1.57\% & -26.79\% \\
09/2020 &  2.24\% &  16.47\% & -1.21\% &   5.56\% \\
10/2020 &  1.30\% &  -4.01\% & -0.98\% & -17.64\% \\
\bottomrule
\end{tabular}
\fdadospesquisa
\end{table}

\begin{table}[htb]
\centering
\caption{Variação das métricas de rede a cada trimestre para o grafo de CNAEs (Parte 1)}
\label{tab:metricas-redes-pandemia:grafo-trimestral-por-cnae1}
\begin{tabular}{l|rrrr}
\toprule
Mês & Densidade & Transitividade & Grau médio & \shortstack{Grau médio\\ponderado} \\
\midrule
2020/1 & - & - & - & - \\
2020/2 & -4.91\% & -1.09\% & -4.91\% & -14.60\% \\
2020/3 &  0.82\% &  0.98\% &  0.91\% &  20.25\% \\
\bottomrule
\end{tabular}
\fdadospesquisa
\end{table}

\begin{table}[htb]
\centering
\caption{Variação das métricas de rede a cada trimestre para o grafo de CNAEs (Parte 2)}
\label{tab:metricas-redes-pandemia:grafo-trimestral-por-cnae2}
\begin{tabular}{l|rrrr}
\toprule
Mês & \shortstack{Assortatividade\\de grau} & \shortstack{Assortatividade\\de seção} & \shortstack{Caminho mínimo\\médio} & \shortstack{Caminho mínimo\\médio ponderado} \\
\midrule
2020/1 & - & - & - & - \\
2020/2 & -0.21\% & 8.07\% & 0.98\% & 71.35\% \\
2020/3 &  0.66\% & 3.81\% & 0.05\% & 26.02\% \\
\bottomrule
\end{tabular}
\fdadospesquisa
\end{table}

\begin{table}[htb]
\centering
\caption{Diferença entre as métricas de rede para CNAEs afetadas e não afetadas}
\label{tab:metricas-redes-pandemia:diferenca-afetadas-por-cnae}
\begin{tabular}{l|rrrrrrr}
\toprule
Métrica & Média & \shortstack{Desvio\\padrão} & Mínimo & P25 & Mediana & P75 & Máximo \\
\midrule
Grau médio                     &  0.02\% & -0.04\% & 0.11\% & 0.05\% &  0.00\% &  0.00\% & -0.05\% \\ \hline
\shortstack[l]{Coeficiente de\\agrupamento}     &  0.16\% & -0.01\% & 0.38\% & 0.15\% &  0.11\% &  0.14\% &  0.07\% \\ \hline
Excentricidade                 &  0.00\% &  0.00\% & 0.00\% & 0.00\% &  0.00\% &  0.00\% &  0.00\% \\ \hline
\shortstack[l]{Caminho mínimo\\médio}           & -0.01\% & -0.03\% & 0.05\% & 0.00\% &  0.00\% & -0.04\% & -0.05\% \\ \hline
\shortstack[l]{Caminho mínimo\\médio ponderado} &  0.00\% & -0.00\% & 0.00\% & 0.00\% & -0.00\% & -0.00\% & -0.00\% \\ \hline
\shortstack[l]{Centralidade de\\proximidade}    &  0.01\% & -0.03\% & 0.05\% & 0.04\% &  0.00\% &  0.00\% & -0.05\% \\ \hline
\shortstack[l]{Centralidade de\\informação}     &  0.00\% & -0.00\% & 0.00\% & 0.00\% &  0.00\% & -0.00\% & -0.00\% \\ \hline
\shortstack[l]{Centralidade de\\autovalor}      &  0.32\% & -0.00\% & 0.56\% & 0.39\% &  0.20\% &  0.25\% &  0.28\% \\ \hline
\shortstack[l]{Centralidade de\\intermediação}  &  0.43\% &  0.12\% & 0.76\% & 0.34\% &  0.00\% &  0.46\% &  0.00\% \\ \hline
PageRank                       &  0.07\% & -0.04\% & 0.17\% & 0.07\% &  0.08\% &  0.11\% & -0.21\% \\
\bottomrule
\end{tabular}
\fdadospesquisa
\end{table}

\section{Detecção de impacto econômico}
\label{section:deteccao-impacto}

\subsection{Dados de UFs}
\label{section:deteccao-impacto:uf}

\begin{figure}[htb] 
    \centering 
    \caption{PCA sobre dados mensais de UFs}
    \label{fig:resultados:base-de-dados-24-pca-monthly-uf} 
    \begin{subfigure}[b]{0.45\textwidth}
        \includegraphics[scale=0.45]{images/base-de-dados-24.1-pca-components-monthly-uf.png}
        \caption{Percentual de variância explicada de cada componente principal}
        \label{fig:resultados:base-de-dados-24.1-pca-components-monthly-uf}
    \end{subfigure} ~ \quad
    \begin{subfigure}[b]{0.45\textwidth}
        \includegraphics[scale=0.45]{images/base-de-dados-24.2-pca-2d-monthly-uf.png}
        \caption{Visualização dos dados projetados sobre as duas componentes principais}
        \label{fig:resultados:base-de-dados-24.2-pca-2d-monthly-uf}
    \end{subfigure}
    \fdadospesquisa
\end{figure}

\begin{figure}[htb] 
    \centering 
    \caption{PCA sobre dados totais de UFs}
    \label{fig:resultados:base-de-dados-25-pca-2d-total-uf} 
    \begin{subfigure}[b]{0.45\textwidth}
        \includegraphics[scale=0.45]{images/base-de-dados-25.1-pca-components-total-uf.png}
        \caption{Percentual de variância explicada de cada componente principal}
        \label{fig:resultados:base-de-dados-25.1-pca-components-total-uf}
    \end{subfigure} ~ \quad
    \begin{subfigure}[b]{0.45\textwidth}
        \includegraphics[scale=0.45]{images/base-de-dados-25.2-pca-2d-total-uf.png}
        \caption{Visualização dos dados projetados sobre as duas componentes principais}
        \label{fig:resultados:base-de-dados-25.2-pca-2d-total-uf}
    \end{subfigure}
    \fdadospesquisa
\end{figure}

\subsection{Dados de Seções}
\label{section:deteccao-impacto:secao}

\begin{figure}[htb] 
    \centering 
    \caption{PCA sobre dados mensais de seções}
    \label{fig:resultados:base-de-dados-26-pca-components-monthly-secao} 
    \begin{subfigure}[b]{0.45\textwidth}
        \includegraphics[scale=0.45]{images/base-de-dados-26.1-pca-components-monthly-secao.png}
        \caption{Percentual de variância explicada de cada componente principal}
        \label{fig:resultados:26.1-pca-components-monthly-seca}
    \end{subfigure} ~ \quad
    \begin{subfigure}[b]{0.45\textwidth}
        \includegraphics[scale=0.45]{images/base-de-dados-26.2-pca-2d-monthly-secao.png}
        \caption{Visualização dos dados projetados sobre as duas componentes principais}
        \label{fig:resultados:base-de-dados-26.2-pca-2d-monthly-secao}
    \end{subfigure}
    \fdadospesquisa
\end{figure}

\begin{figure}[htb] 
    \centering 
    \caption{PCA sobre dados totais de seções}
    \label{fig:resultados:base-de-dados-27-pca-components-total-secao} 
    \begin{subfigure}[b]{0.45\textwidth}
        \includegraphics[scale=0.45]{images/base-de-dados-27.1-pca-components-total-secao.png}
        \caption{Percentual de variância explicada de cada componente principal}
        \label{fig:resultados:base-de-dados-27.1-pca-components-total-secao}
    \end{subfigure} ~ \quad
    \begin{subfigure}[b]{0.45\textwidth}
        \includegraphics[scale=0.45]{images/base-de-dados-27.2-pca-2d-total-secao.png}
        \caption{Visualização dos dados projetados sobre as duas componentes principais}
        \label{fig:resultados:base-de-dados-27.2-pca-2d-total-secao}
    \end{subfigure}
    \fdadospesquisa
\end{figure}

\subsection{Dados de CNAEs}
\label{section:deteccao-impacto:cnae}

\begin{figure}[htb] 
    \centering 
    \caption{PCA sobre dados mensais de CNAEs}
    \label{fig:resultados:base-de-dados-28-pca-components-monthly-cnae} 
    \begin{subfigure}[b]{0.45\textwidth}
        \includegraphics[scale=0.45]{images/base-de-dados-28.1-pca-components-monthly-cnae.png}
        \caption{Percentual de variância explicada de cada componente principal}
        \label{fig:resultados:base-de-dados-28.1-pca-components-monthly-cnae}
    \end{subfigure} ~ \quad
    \begin{subfigure}[b]{0.45\textwidth}
        \includegraphics[scale=0.45]{images/base-de-dados-28.2-pca-2d-monthly-cnae.png}
        \caption{Visualização dos dados projetados sobre as duas componentes principais}
        \label{fig:resultados:base-de-dados-28.2-pca-2d-monthly-cnae}
    \end{subfigure}
    \fdadospesquisa
\end{figure}

\begin{figure}[htb] 
    \centering 
    \caption{PCA sobre dados totais de CNAEs}
    \label{fig:resultados:base-de-dados-28-pca-components-total-cnae} 
    \begin{subfigure}[b]{0.45\textwidth}
        \includegraphics[scale=0.45]{images/base-de-dados-28.1-pca-components-total-cnae.png}
        \caption{Percentual de variância explicada de cada componente principal}
        \label{fig:resultados:base-de-dados-28.1-pca-components-total-cnae}
    \end{subfigure} ~ \quad
    \begin{subfigure}[b]{0.45\textwidth}
        \includegraphics[scale=0.45]{images/base-de-dados-28.2-pca-2d-total-cnae.png}
        \caption{Visualização dos dados projetados sobre as duas componentes principais}
        \label{fig:resultados:base-de-dados-28.2-pca-2d-total-cnae}
    \end{subfigure}
    \fdadospesquisa
\end{figure}

\begin{figure}[htb] 
    \centering 
    \caption{Resultado tentativas de classificação para cada método aplicado sobre dados mensais de CNAE}
    \label{fig:resultados:classification-monthly-cnae} 
    \begin{subfigure}[b]{0.45\textwidth}
        \includegraphics[scale=0.45]{images/base-de-dados-28.3.1-confusion-matrix-randomforest-monthly-cnae.png}
        \caption{Matriz de confusão para Random Forest}
        \label{fig:resultadosbase-de-dados-28.3.1-confusion-matrix-randomforest-monthly-cnae}
    \end{subfigure} ~ \quad
    \begin{subfigure}[b]{0.45\textwidth}
        \includegraphics[scale=0.45]{images/base-de-dados-28.3.2-roc-curve-randomforest-monthly-cnae.png}
        \caption{Curva ROC para Random Forest}
        \label{fig:resultados:base-de-dados-28.3.2-roc-curve-randomforest-monthly-cnae}
    \end{subfigure} ~ \\
    \centering 
    \begin{subfigure}[b]{0.45\textwidth}
        \includegraphics[scale=0.45]{images/base-de-dados-28.3.3-confusion-matrix-svc-monthly-cnae.png}
        \caption{Matriz de confusão para SVC}
        \label{fig:resultados:base-de-dados-28.3.3-confusion-matrix-svc-monthly-cnae}
    \end{subfigure} ~ \quad
    \begin{subfigure}[b]{0.45\textwidth}
        \includegraphics[scale=0.45]{images/base-de-dados-28.3.4-roc-curve-svc-monthly-cnae.png}
        \caption{Curva ROC para SVC}
        \label{fig:resultados:base-de-dados-28.3.4-roc-curve-svc-monthly-cnae}
    \end{subfigure} ~ \\
    \centering
    \begin{subfigure}[b]{0.45\textwidth}
        \includegraphics[scale=0.45]{images/base-de-dados-28.3.5-confusion-matrix-knn-monthly-cnae.png}
        \caption{Matriz de confusão para KNN}
        \label{fig:resultados:base-de-dados-28.3.5-confusion-matrix-knn-monthly-cnae}
    \end{subfigure} ~ \quad
    \begin{subfigure}[b]{0.45\textwidth}
        \includegraphics[scale=0.45]{images/base-de-dados-28.3.6-roc-curve-knn-monthly-cnae.png}
        \caption{Curva ROC para KNN}
        \label{fig:resultados:base-de-dados-28.3.6-roc-curve-knn-monthly-cnae}
    \end{subfigure} ~ \\
    \centering 
    \begin{subfigure}[b]{0.45\textwidth}
        \includegraphics[scale=0.45]{images/base-de-dados-28.3.7-confusion-matrix-logregression-monthly-cnae.png}
        \caption{Matriz de confusão para Regressão logística}
        \label{fig:resultados:base-de-dados-28.3.7-confusion-matrix-logregression-monthly-cnae}
    \end{subfigure} ~ \quad
    \begin{subfigure}[b]{0.45\textwidth}
        \includegraphics[scale=0.45]{images/base-de-dados-28.3.8-roc-curve-logregression-monthly-cnae.png}
        \caption{Curva ROC para Regressão Logística}
        \label{fig:resultados:base-de-dados-28.3.8-roc-curve-logregression-monthly-cnae}
    \end{subfigure} ~ \\
    \fdadospesquisa
\end{figure}

\begin{figure}[htb] 
    \centering 
    \caption{Resultado tentativas de classificação para cada método aplicado sobre dados totais de CNAE}
    \label{fig:resultados:classification-total-nae} 
    \begin{subfigure}[b]{0.45\textwidth}
        \includegraphics[scale=0.45]{images/base-de-dados-28.4.1-confusion-matrix-randomforest-total-cnae.png}
        \caption{Matriz de confusão para Random Forest}
        \label{fig:resultadosbase-de-dados-28.3.1-confusion-matrix-randomforest-total-cnae}
    \end{subfigure} ~ \quad
    \begin{subfigure}[b]{0.45\textwidth}
        \includegraphics[scale=0.45]{images/base-de-dados-28.4.2-roc-curve-randomforest-total-cnae.png}
        \caption{Curva ROC para Random Forest}
        \label{fig:resultados:base-de-dados-28.3.2-roc-curve-randomforest-total-cnae}
    \end{subfigure} ~ \\
    \centering 
    \begin{subfigure}[b]{0.45\textwidth}
        \includegraphics[scale=0.45]{images/base-de-dados-28.4.3-confusion-matrix-svc-total-cnae.png}
        \caption{Matriz de confusão para SVC}
        \label{fig:resultados:base-de-dados-28.3.3-confusion-matrix-svc-total-cnae}
    \end{subfigure} ~ \quad
    \begin{subfigure}[b]{0.45\textwidth}
        \includegraphics[scale=0.45]{images/base-de-dados-28.4.4-roc-curve-svc-total-cnae.png}
        \caption{Curva ROC para SVC}
        \label{fig:resultados:base-de-dados-28.3.4-roc-curve-svc-total-cnae}
    \end{subfigure} ~ \\
    \centering
    \begin{subfigure}[b]{0.45\textwidth}
        \includegraphics[scale=0.45]{images/base-de-dados-28.4.5-confusion-matrix-knn-total-cnae.png}
        \caption{Matriz de confusão para KNN}
        \label{fig:resultados:base-de-dados-28.3.5-confusion-matrix-knn-total-cnae}
    \end{subfigure} ~ \quad
    \begin{subfigure}[b]{0.45\textwidth}
        \includegraphics[scale=0.45]{images/base-de-dados-28.4.6-roc-curve-knn-total-cnae.png}
        \caption{Curva ROC para KNN}
        \label{fig:resultados:base-de-dados-28.3.6-roc-curve-knn-total-cnae}
    \end{subfigure} ~ \\
    \centering 
    \begin{subfigure}[b]{0.45\textwidth}
        \includegraphics[scale=0.45]{images/base-de-dados-28.4.7-confusion-matrix-logregression-total-cnae.png}
        \caption{Matriz de confusão para Regressão logística}
        \label{fig:resultados:base-de-dados-28.3.7-confusion-matrix-logregression-total-cnae}
    \end{subfigure} ~ \quad
    \begin{subfigure}[b]{0.45\textwidth}
        \includegraphics[scale=0.45]{images/base-de-dados-28.4.8-roc-curve-logregression-total-cnae.png}
        \caption{Curva ROC para Regressão Logística}
        \label{fig:resultados:base-de-dados-28.3.8-roc-curve-logregression-total-cnae}
    \end{subfigure} ~ \\
    \fdadospesquisa
\end{figure}

\section{Conclusões}
\label{section:resultados:conclusoes}


\chapter{Conclusão}
\label{chapter:conclusão}

\section{Pontos positivos e negativos}

pros:

métricas de redes contribuíram positivamente para a identificação de crises tanto em um grafo simples (UF) quanto mais complexos (CNAE)

cons:

aplicacao sobre dados de empresas nao foi possivel

\section{Próximos passos}

usar outros documentos fiscais

uso de algoritmos mais escaláveis ou processamento distribuido


% ---
% Finaliza a parte no bookmark do PDF, para que se inicie o bookmark na raiz
% ---
\bookmarksetup{startatroot}% 
% ---

% ----------------------------------------------------------
% ELEMENTOS PÓS-TEXTUAIS
% ----------------------------------------------------------
\postextual

% ----------------------------------------------------------
% Referências bibliográficas
% ----------------------------------------------------------
\bibliography{references}

% ---------------------------------------------------------------------
% GLOSSÁRIO
% ---------------------------------------------------------------------

% Arquivo que contém as definições que vão aparecer no glossário

\newword{Armazém de Dados}{ou \textit{Data Warehouse}, é um sistema de armazenamento estruturado de informações usado para a consolidação de informações e conceitos, geralmente envolvendo uma grande quantidade de dados. Esses dados são comumente armazenados em bancos de dados colunares pois envolvem o cálculo de agregações, e usam técnicas de Processamento Analítico em Tempo Real, também chamado \sigla{OLAP}{Online Analytical Processing}, para a coleta e tratamento de tais dados. Os dados são então usados em processos corporativos para a tomada de decisão, desenvolvimento e aperfeiçoamento de novos produtos, e até mesmo reaproveitados em aplicações}

\newword{\textit{Enterprise Resource Planning}}{ou \textit{ERP}, em tradução livre significa Planejamento de Recursos Empresariais, é um tipo de sistema para o gerenciamento de processos de negócio que podem atender diversas áreas, como financeiro, recursos humanos, compras e vendas, etc. No contexto de documentos fiscais, muitos desses sistemas possuem emissores capazes de gerenciar algumas das obrigações contábeis de forma automatizada}

\newword{Fonte Única de Verdade}{é uma prática adotada na construção de sistemas onde todos os dados conceitualmente relacionados a uma entidade importante para uma empresa ou organização são centralizados em uma única base de dados que passa a se tornar um banco de referência a outros sistemas, aumentando a consistência dos dados e diminuindo a carga cognitiva necessária para times indiretamente ligados a essa entidade}

\newword{\textit{Application Programming Interface}}{ou \textit{API}, em tradução livre significa Interface Programática de Aplicação, é um conjunto de ferramentas que permite a integração entre dois sistemas através de um protocolo de comunicação. Nesta integração são usadas definições comuns a ambos sistemas de modo a estabelecer um padrão de comunicação. APIs geralmente são construídas usando um protocolo de transferência de dados, como HTTP, REST, ou SOAP, e encapsulam o acesso a um banco de dados interno}

% Comando para incluir todas as definições do arquivo glossario.tex
\glsaddall
% Impressão do glossário
\printglossaries

% ----------------------------------------------------------
% Apêndices
% ----------------------------------------------------------

% ---
% Inicia os apêndices
% ---
\begin{apendicesenv}

    \chapter{Armazém de Dados}
    \label{chapter:armazem-de-dados}
    
descricao da arquitetura do data warehouse da arquivei

dfes e dados abertos

    
    \chapter{Tecnologias Utilizadas}
    \label{chapter:tecnologias-utilizadas}
    
Serão descritas aqui algumas das tecnologias utilizadas na execução deste trabalho.

\section{Google BigQuery}

O \textit{Google BigQuery} é um produto disponível na plataforma de nuvem \textit{Google Cloud Platform} \cite{bigquery}, sendo uma tecnologia de armazém de dados gerenciada e altamente escalável. A interface de interação com os dados é a linguagem SQL. Se trata de um banco colunar capaz de lidar com dados da ordem de \textit{petabytes} de dados, e conta com ferramentas de segurança e criptografia nativas.

\begin{figure}[htb]
    \centering
    \caption{Interface do Google BigQuery para seleção de dados}
    \label{fig:bigquery}
    \includegraphics[scale=0.8]{images/bigquery.jpg}
    \fautor
\end{figure}

O \textit{Google BigQuery} é a tecnologia usada no armazenamento de dados do armazém de dados da empresa parceira.

\section{Google AI Platform Notebooks}

Outro produto disponível na \textit{Google Cloud Platform}, o \textit{Google AI Platform Notebooks} \cite{google-notebooks}, fornece um serviço gerenciado de integração segura com o ambiente JupyterLab, uma plataforma \textit{web} para aplicações usando Jupyter Notebook, uma aplicação \textit{web} para a criação e compartilhamento de códigos Python.

\section{Bibliotecas}

Neste trabalho foi utilizada a linguagem Python, com o auxílio das bibliotecas:

\begin{itemize}
    \item \textit{matplotlib} \cite{matplotlib}: para a visualização e criação de imagens.
    \item \textit{sklearn} \cite{sklearn}: para pré-processamento e execução de algoritmos de análise de dados e aprendizado de máquina.
    \item \textit{networkx} \cite{networkx}: para execução de algoritmos de processamento de redes.
    \item \textit{pandarallel} \cite{pandarallel}: para paralelização usando processamento \textit{multithread} para melhor escalabilidade.
\end{itemize}

    
    \chapter{CFOP - Código Financeiro de Operações}
    \label{chapter:cfop}
    
descrição do que é CFOP e quais foram utilizados e excluídos


\end{apendicesenv}
% ---


% ----------------------------------------------------------
% Anexos
% ----------------------------------------------------------

% ---
% Inicia os anexos
% ---
\begin{anexosenv}

    \chapter{Trabalhos Relacionados Relevantes}
    \label{chapter:trabalhos-relevantes}
    
\begin{description}
 \item[\url{http://www.google.com}] Algum link relevante
\end{description}


\end{anexosenv}
% ---

\end{document}